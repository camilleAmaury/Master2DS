\documentclass[11pt]{article}

    \usepackage[breakable]{tcolorbox}
    \usepackage{parskip} % Stop auto-indenting (to mimic markdown behaviour)
    
    \usepackage{iftex}
    \ifPDFTeX
    	\usepackage[T1]{fontenc}
    	\usepackage{mathpazo}
    \else
    	\usepackage{fontspec}
    \fi

    % Basic figure setup, for now with no caption control since it's done
    % automatically by Pandoc (which extracts ![](path) syntax from Markdown).
    \usepackage{graphicx}
    % Maintain compatibility with old templates. Remove in nbconvert 6.0
    \let\Oldincludegraphics\includegraphics
    % Ensure that by default, figures have no caption (until we provide a
    % proper Figure object with a Caption API and a way to capture that
    % in the conversion process - todo).
    \usepackage{caption}
    \DeclareCaptionFormat{nocaption}{}
    \captionsetup{format=nocaption,aboveskip=0pt,belowskip=0pt}

    \usepackage{float}
    \floatplacement{figure}{H} % forces figures to be placed at the correct location
    \usepackage{xcolor} % Allow colors to be defined
    \usepackage{enumerate} % Needed for markdown enumerations to work
    \usepackage{geometry} % Used to adjust the document margins
    \usepackage{amsmath} % Equations
    \usepackage{amssymb} % Equations
    \usepackage{textcomp} % defines textquotesingle
    % Hack from http://tex.stackexchange.com/a/47451/13684:
    \AtBeginDocument{%
        \def\PYZsq{\textquotesingle}% Upright quotes in Pygmentized code
    }
    \usepackage{upquote} % Upright quotes for verbatim code
    \usepackage{eurosym} % defines \euro
    \usepackage[mathletters]{ucs} % Extended unicode (utf-8) support
    \usepackage{fancyvrb} % verbatim replacement that allows latex
    \usepackage{grffile} % extends the file name processing of package graphics 
                         % to support a larger range
    \makeatletter % fix for old versions of grffile with XeLaTeX
    \@ifpackagelater{grffile}{2019/11/01}
    {
      % Do nothing on new versions
    }
    {
      \def\Gread@@xetex#1{%
        \IfFileExists{"\Gin@base".bb}%
        {\Gread@eps{\Gin@base.bb}}%
        {\Gread@@xetex@aux#1}%
      }
    }
    \makeatother
    \usepackage[Export]{adjustbox} % Used to constrain images to a maximum size
    \adjustboxset{max size={0.9\linewidth}{0.9\paperheight}}

    % The hyperref package gives us a pdf with properly built
    % internal navigation ('pdf bookmarks' for the table of contents,
    % internal cross-reference links, web links for URLs, etc.)
    \usepackage{hyperref}
    % The default LaTeX title has an obnoxious amount of whitespace. By default,
    % titling removes some of it. It also provides customization options.
    \usepackage{titling}
    \usepackage{longtable} % longtable support required by pandoc >1.10
    \usepackage{booktabs}  % table support for pandoc > 1.12.2
    \usepackage[inline]{enumitem} % IRkernel/repr support (it uses the enumerate* environment)
    \usepackage[normalem]{ulem} % ulem is needed to support strikethroughs (\sout)
                                % normalem makes italics be italics, not underlines
    \usepackage{mathrsfs}
    

    
    % Colors for the hyperref package
    \definecolor{urlcolor}{rgb}{0,.145,.698}
    \definecolor{linkcolor}{rgb}{.71,0.21,0.01}
    \definecolor{citecolor}{rgb}{.12,.54,.11}

    % ANSI colors
    \definecolor{ansi-black}{HTML}{3E424D}
    \definecolor{ansi-black-intense}{HTML}{282C36}
    \definecolor{ansi-red}{HTML}{E75C58}
    \definecolor{ansi-red-intense}{HTML}{B22B31}
    \definecolor{ansi-green}{HTML}{00A250}
    \definecolor{ansi-green-intense}{HTML}{007427}
    \definecolor{ansi-yellow}{HTML}{DDB62B}
    \definecolor{ansi-yellow-intense}{HTML}{B27D12}
    \definecolor{ansi-blue}{HTML}{208FFB}
    \definecolor{ansi-blue-intense}{HTML}{0065CA}
    \definecolor{ansi-magenta}{HTML}{D160C4}
    \definecolor{ansi-magenta-intense}{HTML}{A03196}
    \definecolor{ansi-cyan}{HTML}{60C6C8}
    \definecolor{ansi-cyan-intense}{HTML}{258F8F}
    \definecolor{ansi-white}{HTML}{C5C1B4}
    \definecolor{ansi-white-intense}{HTML}{A1A6B2}
    \definecolor{ansi-default-inverse-fg}{HTML}{FFFFFF}
    \definecolor{ansi-default-inverse-bg}{HTML}{000000}

    % common color for the border for error outputs.
    \definecolor{outerrorbackground}{HTML}{FFDFDF}

    % commands and environments needed by pandoc snippets
    % extracted from the output of `pandoc -s`
    \providecommand{\tightlist}{%
      \setlength{\itemsep}{0pt}\setlength{\parskip}{0pt}}
    \DefineVerbatimEnvironment{Highlighting}{Verbatim}{commandchars=\\\{\}}
    % Add ',fontsize=\small' for more characters per line
    \newenvironment{Shaded}{}{}
    \newcommand{\KeywordTok}[1]{\textcolor[rgb]{0.00,0.44,0.13}{\textbf{{#1}}}}
    \newcommand{\DataTypeTok}[1]{\textcolor[rgb]{0.56,0.13,0.00}{{#1}}}
    \newcommand{\DecValTok}[1]{\textcolor[rgb]{0.25,0.63,0.44}{{#1}}}
    \newcommand{\BaseNTok}[1]{\textcolor[rgb]{0.25,0.63,0.44}{{#1}}}
    \newcommand{\FloatTok}[1]{\textcolor[rgb]{0.25,0.63,0.44}{{#1}}}
    \newcommand{\CharTok}[1]{\textcolor[rgb]{0.25,0.44,0.63}{{#1}}}
    \newcommand{\StringTok}[1]{\textcolor[rgb]{0.25,0.44,0.63}{{#1}}}
    \newcommand{\CommentTok}[1]{\textcolor[rgb]{0.38,0.63,0.69}{\textit{{#1}}}}
    \newcommand{\OtherTok}[1]{\textcolor[rgb]{0.00,0.44,0.13}{{#1}}}
    \newcommand{\AlertTok}[1]{\textcolor[rgb]{1.00,0.00,0.00}{\textbf{{#1}}}}
    \newcommand{\FunctionTok}[1]{\textcolor[rgb]{0.02,0.16,0.49}{{#1}}}
    \newcommand{\RegionMarkerTok}[1]{{#1}}
    \newcommand{\ErrorTok}[1]{\textcolor[rgb]{1.00,0.00,0.00}{\textbf{{#1}}}}
    \newcommand{\NormalTok}[1]{{#1}}
    
    % Additional commands for more recent versions of Pandoc
    \newcommand{\ConstantTok}[1]{\textcolor[rgb]{0.53,0.00,0.00}{{#1}}}
    \newcommand{\SpecialCharTok}[1]{\textcolor[rgb]{0.25,0.44,0.63}{{#1}}}
    \newcommand{\VerbatimStringTok}[1]{\textcolor[rgb]{0.25,0.44,0.63}{{#1}}}
    \newcommand{\SpecialStringTok}[1]{\textcolor[rgb]{0.73,0.40,0.53}{{#1}}}
    \newcommand{\ImportTok}[1]{{#1}}
    \newcommand{\DocumentationTok}[1]{\textcolor[rgb]{0.73,0.13,0.13}{\textit{{#1}}}}
    \newcommand{\AnnotationTok}[1]{\textcolor[rgb]{0.38,0.63,0.69}{\textbf{\textit{{#1}}}}}
    \newcommand{\CommentVarTok}[1]{\textcolor[rgb]{0.38,0.63,0.69}{\textbf{\textit{{#1}}}}}
    \newcommand{\VariableTok}[1]{\textcolor[rgb]{0.10,0.09,0.49}{{#1}}}
    \newcommand{\ControlFlowTok}[1]{\textcolor[rgb]{0.00,0.44,0.13}{\textbf{{#1}}}}
    \newcommand{\OperatorTok}[1]{\textcolor[rgb]{0.40,0.40,0.40}{{#1}}}
    \newcommand{\BuiltInTok}[1]{{#1}}
    \newcommand{\ExtensionTok}[1]{{#1}}
    \newcommand{\PreprocessorTok}[1]{\textcolor[rgb]{0.74,0.48,0.00}{{#1}}}
    \newcommand{\AttributeTok}[1]{\textcolor[rgb]{0.49,0.56,0.16}{{#1}}}
    \newcommand{\InformationTok}[1]{\textcolor[rgb]{0.38,0.63,0.69}{\textbf{\textit{{#1}}}}}
    \newcommand{\WarningTok}[1]{\textcolor[rgb]{0.38,0.63,0.69}{\textbf{\textit{{#1}}}}}
    
    
    % Define a nice break command that doesn't care if a line doesn't already
    % exist.
    \def\br{\hspace*{\fill} \\* }
    % Math Jax compatibility definitions
    \def\gt{>}
    \def\lt{<}
    \let\Oldtex\TeX
    \let\Oldlatex\LaTeX
    \renewcommand{\TeX}{\textrm{\Oldtex}}
    \renewcommand{\LaTeX}{\textrm{\Oldlatex}}
    % Document parameters
    % Document title
    \title{Report Autonomous Work 2}
    
    
    
    
    
% Pygments definitions
\makeatletter
\def\PY@reset{\let\PY@it=\relax \let\PY@bf=\relax%
    \let\PY@ul=\relax \let\PY@tc=\relax%
    \let\PY@bc=\relax \let\PY@ff=\relax}
\def\PY@tok#1{\csname PY@tok@#1\endcsname}
\def\PY@toks#1+{\ifx\relax#1\empty\else%
    \PY@tok{#1}\expandafter\PY@toks\fi}
\def\PY@do#1{\PY@bc{\PY@tc{\PY@ul{%
    \PY@it{\PY@bf{\PY@ff{#1}}}}}}}
\def\PY#1#2{\PY@reset\PY@toks#1+\relax+\PY@do{#2}}

\expandafter\def\csname PY@tok@w\endcsname{\def\PY@tc##1{\textcolor[rgb]{0.73,0.73,0.73}{##1}}}
\expandafter\def\csname PY@tok@c\endcsname{\let\PY@it=\textit\def\PY@tc##1{\textcolor[rgb]{0.25,0.50,0.50}{##1}}}
\expandafter\def\csname PY@tok@cp\endcsname{\def\PY@tc##1{\textcolor[rgb]{0.74,0.48,0.00}{##1}}}
\expandafter\def\csname PY@tok@k\endcsname{\let\PY@bf=\textbf\def\PY@tc##1{\textcolor[rgb]{0.00,0.50,0.00}{##1}}}
\expandafter\def\csname PY@tok@kp\endcsname{\def\PY@tc##1{\textcolor[rgb]{0.00,0.50,0.00}{##1}}}
\expandafter\def\csname PY@tok@kt\endcsname{\def\PY@tc##1{\textcolor[rgb]{0.69,0.00,0.25}{##1}}}
\expandafter\def\csname PY@tok@o\endcsname{\def\PY@tc##1{\textcolor[rgb]{0.40,0.40,0.40}{##1}}}
\expandafter\def\csname PY@tok@ow\endcsname{\let\PY@bf=\textbf\def\PY@tc##1{\textcolor[rgb]{0.67,0.13,1.00}{##1}}}
\expandafter\def\csname PY@tok@nb\endcsname{\def\PY@tc##1{\textcolor[rgb]{0.00,0.50,0.00}{##1}}}
\expandafter\def\csname PY@tok@nf\endcsname{\def\PY@tc##1{\textcolor[rgb]{0.00,0.00,1.00}{##1}}}
\expandafter\def\csname PY@tok@nc\endcsname{\let\PY@bf=\textbf\def\PY@tc##1{\textcolor[rgb]{0.00,0.00,1.00}{##1}}}
\expandafter\def\csname PY@tok@nn\endcsname{\let\PY@bf=\textbf\def\PY@tc##1{\textcolor[rgb]{0.00,0.00,1.00}{##1}}}
\expandafter\def\csname PY@tok@ne\endcsname{\let\PY@bf=\textbf\def\PY@tc##1{\textcolor[rgb]{0.82,0.25,0.23}{##1}}}
\expandafter\def\csname PY@tok@nv\endcsname{\def\PY@tc##1{\textcolor[rgb]{0.10,0.09,0.49}{##1}}}
\expandafter\def\csname PY@tok@no\endcsname{\def\PY@tc##1{\textcolor[rgb]{0.53,0.00,0.00}{##1}}}
\expandafter\def\csname PY@tok@nl\endcsname{\def\PY@tc##1{\textcolor[rgb]{0.63,0.63,0.00}{##1}}}
\expandafter\def\csname PY@tok@ni\endcsname{\let\PY@bf=\textbf\def\PY@tc##1{\textcolor[rgb]{0.60,0.60,0.60}{##1}}}
\expandafter\def\csname PY@tok@na\endcsname{\def\PY@tc##1{\textcolor[rgb]{0.49,0.56,0.16}{##1}}}
\expandafter\def\csname PY@tok@nt\endcsname{\let\PY@bf=\textbf\def\PY@tc##1{\textcolor[rgb]{0.00,0.50,0.00}{##1}}}
\expandafter\def\csname PY@tok@nd\endcsname{\def\PY@tc##1{\textcolor[rgb]{0.67,0.13,1.00}{##1}}}
\expandafter\def\csname PY@tok@s\endcsname{\def\PY@tc##1{\textcolor[rgb]{0.73,0.13,0.13}{##1}}}
\expandafter\def\csname PY@tok@sd\endcsname{\let\PY@it=\textit\def\PY@tc##1{\textcolor[rgb]{0.73,0.13,0.13}{##1}}}
\expandafter\def\csname PY@tok@si\endcsname{\let\PY@bf=\textbf\def\PY@tc##1{\textcolor[rgb]{0.73,0.40,0.53}{##1}}}
\expandafter\def\csname PY@tok@se\endcsname{\let\PY@bf=\textbf\def\PY@tc##1{\textcolor[rgb]{0.73,0.40,0.13}{##1}}}
\expandafter\def\csname PY@tok@sr\endcsname{\def\PY@tc##1{\textcolor[rgb]{0.73,0.40,0.53}{##1}}}
\expandafter\def\csname PY@tok@ss\endcsname{\def\PY@tc##1{\textcolor[rgb]{0.10,0.09,0.49}{##1}}}
\expandafter\def\csname PY@tok@sx\endcsname{\def\PY@tc##1{\textcolor[rgb]{0.00,0.50,0.00}{##1}}}
\expandafter\def\csname PY@tok@m\endcsname{\def\PY@tc##1{\textcolor[rgb]{0.40,0.40,0.40}{##1}}}
\expandafter\def\csname PY@tok@gh\endcsname{\let\PY@bf=\textbf\def\PY@tc##1{\textcolor[rgb]{0.00,0.00,0.50}{##1}}}
\expandafter\def\csname PY@tok@gu\endcsname{\let\PY@bf=\textbf\def\PY@tc##1{\textcolor[rgb]{0.50,0.00,0.50}{##1}}}
\expandafter\def\csname PY@tok@gd\endcsname{\def\PY@tc##1{\textcolor[rgb]{0.63,0.00,0.00}{##1}}}
\expandafter\def\csname PY@tok@gi\endcsname{\def\PY@tc##1{\textcolor[rgb]{0.00,0.63,0.00}{##1}}}
\expandafter\def\csname PY@tok@gr\endcsname{\def\PY@tc##1{\textcolor[rgb]{1.00,0.00,0.00}{##1}}}
\expandafter\def\csname PY@tok@ge\endcsname{\let\PY@it=\textit}
\expandafter\def\csname PY@tok@gs\endcsname{\let\PY@bf=\textbf}
\expandafter\def\csname PY@tok@gp\endcsname{\let\PY@bf=\textbf\def\PY@tc##1{\textcolor[rgb]{0.00,0.00,0.50}{##1}}}
\expandafter\def\csname PY@tok@go\endcsname{\def\PY@tc##1{\textcolor[rgb]{0.53,0.53,0.53}{##1}}}
\expandafter\def\csname PY@tok@gt\endcsname{\def\PY@tc##1{\textcolor[rgb]{0.00,0.27,0.87}{##1}}}
\expandafter\def\csname PY@tok@err\endcsname{\def\PY@bc##1{\setlength{\fboxsep}{0pt}\fcolorbox[rgb]{1.00,0.00,0.00}{1,1,1}{\strut ##1}}}
\expandafter\def\csname PY@tok@kc\endcsname{\let\PY@bf=\textbf\def\PY@tc##1{\textcolor[rgb]{0.00,0.50,0.00}{##1}}}
\expandafter\def\csname PY@tok@kd\endcsname{\let\PY@bf=\textbf\def\PY@tc##1{\textcolor[rgb]{0.00,0.50,0.00}{##1}}}
\expandafter\def\csname PY@tok@kn\endcsname{\let\PY@bf=\textbf\def\PY@tc##1{\textcolor[rgb]{0.00,0.50,0.00}{##1}}}
\expandafter\def\csname PY@tok@kr\endcsname{\let\PY@bf=\textbf\def\PY@tc##1{\textcolor[rgb]{0.00,0.50,0.00}{##1}}}
\expandafter\def\csname PY@tok@bp\endcsname{\def\PY@tc##1{\textcolor[rgb]{0.00,0.50,0.00}{##1}}}
\expandafter\def\csname PY@tok@fm\endcsname{\def\PY@tc##1{\textcolor[rgb]{0.00,0.00,1.00}{##1}}}
\expandafter\def\csname PY@tok@vc\endcsname{\def\PY@tc##1{\textcolor[rgb]{0.10,0.09,0.49}{##1}}}
\expandafter\def\csname PY@tok@vg\endcsname{\def\PY@tc##1{\textcolor[rgb]{0.10,0.09,0.49}{##1}}}
\expandafter\def\csname PY@tok@vi\endcsname{\def\PY@tc##1{\textcolor[rgb]{0.10,0.09,0.49}{##1}}}
\expandafter\def\csname PY@tok@vm\endcsname{\def\PY@tc##1{\textcolor[rgb]{0.10,0.09,0.49}{##1}}}
\expandafter\def\csname PY@tok@sa\endcsname{\def\PY@tc##1{\textcolor[rgb]{0.73,0.13,0.13}{##1}}}
\expandafter\def\csname PY@tok@sb\endcsname{\def\PY@tc##1{\textcolor[rgb]{0.73,0.13,0.13}{##1}}}
\expandafter\def\csname PY@tok@sc\endcsname{\def\PY@tc##1{\textcolor[rgb]{0.73,0.13,0.13}{##1}}}
\expandafter\def\csname PY@tok@dl\endcsname{\def\PY@tc##1{\textcolor[rgb]{0.73,0.13,0.13}{##1}}}
\expandafter\def\csname PY@tok@s2\endcsname{\def\PY@tc##1{\textcolor[rgb]{0.73,0.13,0.13}{##1}}}
\expandafter\def\csname PY@tok@sh\endcsname{\def\PY@tc##1{\textcolor[rgb]{0.73,0.13,0.13}{##1}}}
\expandafter\def\csname PY@tok@s1\endcsname{\def\PY@tc##1{\textcolor[rgb]{0.73,0.13,0.13}{##1}}}
\expandafter\def\csname PY@tok@mb\endcsname{\def\PY@tc##1{\textcolor[rgb]{0.40,0.40,0.40}{##1}}}
\expandafter\def\csname PY@tok@mf\endcsname{\def\PY@tc##1{\textcolor[rgb]{0.40,0.40,0.40}{##1}}}
\expandafter\def\csname PY@tok@mh\endcsname{\def\PY@tc##1{\textcolor[rgb]{0.40,0.40,0.40}{##1}}}
\expandafter\def\csname PY@tok@mi\endcsname{\def\PY@tc##1{\textcolor[rgb]{0.40,0.40,0.40}{##1}}}
\expandafter\def\csname PY@tok@il\endcsname{\def\PY@tc##1{\textcolor[rgb]{0.40,0.40,0.40}{##1}}}
\expandafter\def\csname PY@tok@mo\endcsname{\def\PY@tc##1{\textcolor[rgb]{0.40,0.40,0.40}{##1}}}
\expandafter\def\csname PY@tok@ch\endcsname{\let\PY@it=\textit\def\PY@tc##1{\textcolor[rgb]{0.25,0.50,0.50}{##1}}}
\expandafter\def\csname PY@tok@cm\endcsname{\let\PY@it=\textit\def\PY@tc##1{\textcolor[rgb]{0.25,0.50,0.50}{##1}}}
\expandafter\def\csname PY@tok@cpf\endcsname{\let\PY@it=\textit\def\PY@tc##1{\textcolor[rgb]{0.25,0.50,0.50}{##1}}}
\expandafter\def\csname PY@tok@c1\endcsname{\let\PY@it=\textit\def\PY@tc##1{\textcolor[rgb]{0.25,0.50,0.50}{##1}}}
\expandafter\def\csname PY@tok@cs\endcsname{\let\PY@it=\textit\def\PY@tc##1{\textcolor[rgb]{0.25,0.50,0.50}{##1}}}

\def\PYZbs{\char`\\}
\def\PYZus{\char`\_}
\def\PYZob{\char`\{}
\def\PYZcb{\char`\}}
\def\PYZca{\char`\^}
\def\PYZam{\char`\&}
\def\PYZlt{\char`\<}
\def\PYZgt{\char`\>}
\def\PYZsh{\char`\#}
\def\PYZpc{\char`\%}
\def\PYZdl{\char`\$}
\def\PYZhy{\char`\-}
\def\PYZsq{\char`\'}
\def\PYZdq{\char`\"}
\def\PYZti{\char`\~}
% for compatibility with earlier versions
\def\PYZat{@}
\def\PYZlb{[}
\def\PYZrb{]}
\makeatother


    % For linebreaks inside Verbatim environment from package fancyvrb. 
    \makeatletter
        \newbox\Wrappedcontinuationbox 
        \newbox\Wrappedvisiblespacebox 
        \newcommand*\Wrappedvisiblespace {\textcolor{red}{\textvisiblespace}} 
        \newcommand*\Wrappedcontinuationsymbol {\textcolor{red}{\llap{\tiny$\m@th\hookrightarrow$}}} 
        \newcommand*\Wrappedcontinuationindent {3ex } 
        \newcommand*\Wrappedafterbreak {\kern\Wrappedcontinuationindent\copy\Wrappedcontinuationbox} 
        % Take advantage of the already applied Pygments mark-up to insert 
        % potential linebreaks for TeX processing. 
        %        {, <, #, %, $, ' and ": go to next line. 
        %        _, }, ^, &, >, - and ~: stay at end of broken line. 
        % Use of \textquotesingle for straight quote. 
        \newcommand*\Wrappedbreaksatspecials {% 
            \def\PYGZus{\discretionary{\char`\_}{\Wrappedafterbreak}{\char`\_}}% 
            \def\PYGZob{\discretionary{}{\Wrappedafterbreak\char`\{}{\char`\{}}% 
            \def\PYGZcb{\discretionary{\char`\}}{\Wrappedafterbreak}{\char`\}}}% 
            \def\PYGZca{\discretionary{\char`\^}{\Wrappedafterbreak}{\char`\^}}% 
            \def\PYGZam{\discretionary{\char`\&}{\Wrappedafterbreak}{\char`\&}}% 
            \def\PYGZlt{\discretionary{}{\Wrappedafterbreak\char`\<}{\char`\<}}% 
            \def\PYGZgt{\discretionary{\char`\>}{\Wrappedafterbreak}{\char`\>}}% 
            \def\PYGZsh{\discretionary{}{\Wrappedafterbreak\char`\#}{\char`\#}}% 
            \def\PYGZpc{\discretionary{}{\Wrappedafterbreak\char`\%}{\char`\%}}% 
            \def\PYGZdl{\discretionary{}{\Wrappedafterbreak\char`\$}{\char`\$}}% 
            \def\PYGZhy{\discretionary{\char`\-}{\Wrappedafterbreak}{\char`\-}}% 
            \def\PYGZsq{\discretionary{}{\Wrappedafterbreak\textquotesingle}{\textquotesingle}}% 
            \def\PYGZdq{\discretionary{}{\Wrappedafterbreak\char`\"}{\char`\"}}% 
            \def\PYGZti{\discretionary{\char`\~}{\Wrappedafterbreak}{\char`\~}}% 
        } 
        % Some characters . , ; ? ! / are not pygmentized. 
        % This macro makes them "active" and they will insert potential linebreaks 
        \newcommand*\Wrappedbreaksatpunct {% 
            \lccode`\~`\.\lowercase{\def~}{\discretionary{\hbox{\char`\.}}{\Wrappedafterbreak}{\hbox{\char`\.}}}% 
            \lccode`\~`\,\lowercase{\def~}{\discretionary{\hbox{\char`\,}}{\Wrappedafterbreak}{\hbox{\char`\,}}}% 
            \lccode`\~`\;\lowercase{\def~}{\discretionary{\hbox{\char`\;}}{\Wrappedafterbreak}{\hbox{\char`\;}}}% 
            \lccode`\~`\:\lowercase{\def~}{\discretionary{\hbox{\char`\:}}{\Wrappedafterbreak}{\hbox{\char`\:}}}% 
            \lccode`\~`\?\lowercase{\def~}{\discretionary{\hbox{\char`\?}}{\Wrappedafterbreak}{\hbox{\char`\?}}}% 
            \lccode`\~`\!\lowercase{\def~}{\discretionary{\hbox{\char`\!}}{\Wrappedafterbreak}{\hbox{\char`\!}}}% 
            \lccode`\~`\/\lowercase{\def~}{\discretionary{\hbox{\char`\/}}{\Wrappedafterbreak}{\hbox{\char`\/}}}% 
            \catcode`\.\active
            \catcode`\,\active 
            \catcode`\;\active
            \catcode`\:\active
            \catcode`\?\active
            \catcode`\!\active
            \catcode`\/\active 
            \lccode`\~`\~ 	
        }
    \makeatother

    \let\OriginalVerbatim=\Verbatim
    \makeatletter
    \renewcommand{\Verbatim}[1][1]{%
        %\parskip\z@skip
        \sbox\Wrappedcontinuationbox {\Wrappedcontinuationsymbol}%
        \sbox\Wrappedvisiblespacebox {\FV@SetupFont\Wrappedvisiblespace}%
        \def\FancyVerbFormatLine ##1{\hsize\linewidth
            \vtop{\raggedright\hyphenpenalty\z@\exhyphenpenalty\z@
                \doublehyphendemerits\z@\finalhyphendemerits\z@
                \strut ##1\strut}%
        }%
        % If the linebreak is at a space, the latter will be displayed as visible
        % space at end of first line, and a continuation symbol starts next line.
        % Stretch/shrink are however usually zero for typewriter font.
        \def\FV@Space {%
            \nobreak\hskip\z@ plus\fontdimen3\font minus\fontdimen4\font
            \discretionary{\copy\Wrappedvisiblespacebox}{\Wrappedafterbreak}
            {\kern\fontdimen2\font}%
        }%
        
        % Allow breaks at special characters using \PYG... macros.
        \Wrappedbreaksatspecials
        % Breaks at punctuation characters . , ; ? ! and / need catcode=\active 	
        \OriginalVerbatim[#1,codes*=\Wrappedbreaksatpunct]%
    }
    \makeatother

    % Exact colors from NB
    \definecolor{incolor}{HTML}{303F9F}
    \definecolor{outcolor}{HTML}{D84315}
    \definecolor{cellborder}{HTML}{CFCFCF}
    \definecolor{cellbackground}{HTML}{F7F7F7}
    
    % prompt
    \makeatletter
    \newcommand{\boxspacing}{\kern\kvtcb@left@rule\kern\kvtcb@boxsep}
    \makeatother
    \newcommand{\prompt}[4]{
        {\ttfamily\llap{{\color{#2}[#3]:\hspace{3pt}#4}}\vspace{-\baselineskip}}
    }
    

    
    % Prevent overflowing lines due to hard-to-break entities
    \sloppy 
    % Setup hyperref package
    \hypersetup{
      breaklinks=true,  % so long urls are correctly broken across lines
      colorlinks=true,
      urlcolor=urlcolor,
      linkcolor=linkcolor,
      citecolor=citecolor,
      }
    % Slightly bigger margins than the latex defaults
    
    \geometry{verbose,tmargin=1in,bmargin=1in,lmargin=1in,rmargin=1in}
    
    

\begin{document}
    
    \maketitle
    
    

    
    \hypertarget{report-2-clustering}{%
\section{Report 2 : Clustering}\label{report-2-clustering}}

Clément Antheaume, Camille-Amaury Juge.

    \begin{tcolorbox}[breakable, size=fbox, boxrule=1pt, pad at break*=1mm,colback=cellbackground, colframe=cellborder]
\prompt{In}{incolor}{1}{\boxspacing}
\begin{Verbatim}[commandchars=\\\{\}]
\PY{k+kn}{from} \PY{n+nn}{IPython}\PY{n+nn}{.}\PY{n+nn}{display} \PY{k+kn}{import} \PY{n}{Image}
\PY{n}{display}\PY{p}{(}\PY{n}{Image}\PY{p}{(}\PY{n}{filename}\PY{o}{=}\PY{l+s+s1}{\PYZsq{}}\PY{l+s+s1}{images/nantes.png}\PY{l+s+s1}{\PYZsq{}}\PY{p}{)}\PY{p}{)}
\end{Verbatim}
\end{tcolorbox}

    \begin{center}
    \adjustimage{max size={0.9\linewidth}{0.9\paperheight}}{output_1_0.png}
    \end{center}
    { \hspace*{\fill} \\}
    
    \hypertarget{definition-of-indicators}{%
\subsection{Definition of indicators}\label{definition-of-indicators}}

    \hypertarget{continuous}{%
\subsubsection{Continuous}\label{continuous}}

    \hypertarget{silhouette-coefficient}{%
\paragraph{Silhouette coefficient :}\label{silhouette-coefficient}}

    \begin{itemize}
\tightlist
\item
  Average distance of an individual to his suspected class minus the
  average distance with points from other groups.
\item
  Is in the {[}-1; 1{]} scope. If the difference is negative, the point
  is then considered misclassed.
\item
  1 is the best coefficient possible, meaning the point is well classed
  and -1 the worst, meaning the point can't be more misclassed.
\end{itemize}

    \hypertarget{dunn-index}{%
\paragraph{Dunn Index :}\label{dunn-index}}

    \begin{itemize}
\tightlist
\item
  Is equal to the maximun distance between two individuals of the same
  class divided by the minimum distance between two indivudual from
  different classes.
\item
  Any metric can be used, such as Euclidian distance, Manhattan
  distance, Mahalanobis distance\ldots{}
\item
  Is in the {[}0; +infinite{]} scope, with 0 being the worst score
  possible.
\end{itemize}

    \hypertarget{davies-bouldin-index}{%
\paragraph{Davies-Bouldin Index :}\label{davies-bouldin-index}}

    \begin{itemize}
\tightlist
\item
  Is the distance of a individual to the center of its cluster divided
  by the distance between 2 groups' centers
\item
  Is in the {[}0; +infinite{]} scope, with 0 being the best score
  possible.
\end{itemize}

    \hypertarget{discrete}{%
\subsubsection{Discrete}\label{discrete}}

    \hypertarget{modularity}{%
\paragraph{Modularity :}\label{modularity}}

    \begin{itemize}
\tightlist
\item
  Measure the strength of a network/graph division.
\item
  The Modularity matrix is equal to the number of edges between two
  nodes minus the expected number of edges between them, which is equal
  to the product of their degrees divided by 2 times the number of edges
  in the graph.
\item
  Fails to resolve small communities because the optimization part often
  merge small communities together.
\end{itemize}

    \hypertarget{v-measure}{%
\paragraph{V-measure :}\label{v-measure}}

    \begin{itemize}
\tightlist
\item
  Harmonic mean between homogenity and completeness
\item
  Can be computed if you know the labels of the clustered data
\item
  (1+beta) * homogenity * completeness / (beta * homogenity +
  completeness)
\end{itemize}

    \hypertarget{validation-techniques}{%
\subsubsection{Validation Techniques}\label{validation-techniques}}

    \hypertarget{bootstrapping}{%
\paragraph{Bootstrapping}\label{bootstrapping}}

    \begin{itemize}
\tightlist
\item
  The bootstrapping technique is about resampling the data a huge number
  of times. The resampled data is usually the same size of the initial
  data, with a random order and even duplicated data.
\item
  On each sample, several statistics can be calculated, such as mean,
  variance, randindex or distances to the center.
\end{itemize}

    \hypertarget{jack-knife}{%
\paragraph{Jack-knife}\label{jack-knife}}

    \begin{itemize}
\tightlist
\item
  The Jack-knife method is a very versatile method used to every. It
  consists in removing a small sample of the dataset and trying to
  recompute its statistics, just like a the method above.
\end{itemize}

    \hypertarget{external-cluster-validation}{%
\paragraph{External cluster
validation}\label{external-cluster-validation}}

    \begin{itemize}
\tightlist
\item
  Removing the labels of a classification problem, applying a clustering
  algorithm and compare the predicted class of the individuals to the
  removed one.
\end{itemize}

    \hypertarget{python-part}{%
\subsection{Python Part}\label{python-part}}

    \hypertarget{imports}{%
\subsubsection{Imports}\label{imports}}

    \begin{tcolorbox}[breakable, size=fbox, boxrule=1pt, pad at break*=1mm,colback=cellbackground, colframe=cellborder]
\prompt{In}{incolor}{2}{\boxspacing}
\begin{Verbatim}[commandchars=\\\{\}]
\PY{k+kn}{import} \PY{n+nn}{pandas} \PY{k}{as} \PY{n+nn}{pd}
\PY{k+kn}{from} \PY{n+nn}{sklearn}\PY{n+nn}{.}\PY{n+nn}{cluster} \PY{k+kn}{import} \PY{n}{KMeans}
\PY{k+kn}{import} \PY{n+nn}{numpy} \PY{k}{as} \PY{n+nn}{np}
\PY{k+kn}{import} \PY{n+nn}{matplotlib}\PY{n+nn}{.}\PY{n+nn}{pyplot} \PY{k}{as} \PY{n+nn}{plt}
\PY{k+kn}{from} \PY{n+nn}{sklearn} \PY{k+kn}{import} \PY{n}{metrics}
\PY{k+kn}{from} \PY{n+nn}{yellowbrick}\PY{n+nn}{.}\PY{n+nn}{cluster} \PY{k+kn}{import} \PY{n}{KElbowVisualizer}
\end{Verbatim}
\end{tcolorbox}

    \begin{Verbatim}[commandchars=\\\{\}]
C:\textbackslash{}Users\textbackslash{}CAJug\textbackslash{}anaconda3\textbackslash{}envs\textbackslash{}R\_py37\textbackslash{}lib\textbackslash{}site-
packages\textbackslash{}sklearn\textbackslash{}utils\textbackslash{}deprecation.py:143: FutureWarning: The
sklearn.metrics.classification module is  deprecated in version 0.22 and will be
removed in version 0.24. The corresponding classes / functions should instead be
imported from sklearn.metrics. Anything that cannot be imported from
sklearn.metrics is now part of the private API.
  warnings.warn(message, FutureWarning)
    \end{Verbatim}

    \hypertarget{datasets-meteo}{%
\subsubsection{Datasets Meteo}\label{datasets-meteo}}

    \begin{tcolorbox}[breakable, size=fbox, boxrule=1pt, pad at break*=1mm,colback=cellbackground, colframe=cellborder]
\prompt{In}{incolor}{3}{\boxspacing}
\begin{Verbatim}[commandchars=\\\{\}]
\PY{n}{features} \PY{o}{=} \PY{p}{[}\PY{l+s+s1}{\PYZsq{}}\PY{l+s+s1}{Outlook}\PY{l+s+s1}{\PYZsq{}}\PY{p}{,}\PY{l+s+s1}{\PYZsq{}}\PY{l+s+s1}{Tempreature}\PY{l+s+s1}{\PYZsq{}}\PY{p}{,}\PY{l+s+s1}{\PYZsq{}}\PY{l+s+s1}{Humidity}\PY{l+s+s1}{\PYZsq{}}\PY{p}{,}\PY{l+s+s1}{\PYZsq{}}\PY{l+s+s1}{Windy}\PY{l+s+s1}{\PYZsq{}}\PY{p}{]}
\PY{n}{classes} \PY{o}{=} \PY{p}{[}\PY{l+s+s1}{\PYZsq{}}\PY{l+s+s1}{Umbrella}\PY{l+s+s1}{\PYZsq{}}\PY{p}{,}\PY{l+s+s1}{\PYZsq{}}\PY{l+s+s1}{No umbrella}\PY{l+s+s1}{\PYZsq{}}\PY{p}{]}

\PY{n}{dic\PYZus{}X} \PY{o}{=} \PY{p}{\PYZob{}}\PY{l+s+s1}{\PYZsq{}}\PY{l+s+s1}{sunny}\PY{l+s+s1}{\PYZsq{}} \PY{p}{:} \PY{l+m+mi}{0}\PY{p}{,} \PY{l+s+s1}{\PYZsq{}}\PY{l+s+s1}{overcast}\PY{l+s+s1}{\PYZsq{}} \PY{p}{:} \PY{l+m+mi}{1}\PY{p}{,} \PY{l+s+s1}{\PYZsq{}}\PY{l+s+s1}{rain}\PY{l+s+s1}{\PYZsq{}} \PY{p}{:} \PY{l+m+mi}{2}\PY{p}{,} \PY{l+s+s1}{\PYZsq{}}\PY{l+s+s1}{cool}\PY{l+s+s1}{\PYZsq{}} \PY{p}{:} \PY{l+m+mi}{0}\PY{p}{,} \PY{l+s+s1}{\PYZsq{}}\PY{l+s+s1}{mild}\PY{l+s+s1}{\PYZsq{}} \PY{p}{:} \PY{l+m+mi}{1}\PY{p}{,} \PY{l+s+s1}{\PYZsq{}}\PY{l+s+s1}{hot}\PY{l+s+s1}{\PYZsq{}} \PY{p}{:} \PY{l+m+mi}{2}\PY{p}{,} \PY{k+kc}{True} \PY{p}{:} \PY{l+m+mi}{1}\PY{p}{,} \PY{k+kc}{False} \PY{p}{:} \PY{l+m+mi}{0}\PY{p}{,} \PY{l+s+s1}{\PYZsq{}}\PY{l+s+s1}{high}\PY{l+s+s1}{\PYZsq{}} \PY{p}{:} \PY{l+m+mi}{1}\PY{p}{,} \PY{l+s+s1}{\PYZsq{}}\PY{l+s+s1}{normal}\PY{l+s+s1}{\PYZsq{}} \PY{p}{:} \PY{l+m+mi}{0}\PY{p}{\PYZcb{}}
\PY{n}{dic\PYZus{}Y} \PY{o}{=} \PY{p}{\PYZob{}}\PY{l+s+s1}{\PYZsq{}}\PY{l+s+s1}{N}\PY{l+s+s1}{\PYZsq{}} \PY{p}{:} \PY{l+m+mi}{0}\PY{p}{,} \PY{l+s+s1}{\PYZsq{}}\PY{l+s+s1}{P}\PY{l+s+s1}{\PYZsq{}} \PY{p}{:} \PY{l+m+mi}{1}\PY{p}{\PYZcb{}}

\PY{n}{data} \PY{o}{=} \PY{n}{pd}\PY{o}{.}\PY{n}{read\PYZus{}excel}\PY{p}{(}\PY{l+s+s1}{\PYZsq{}}\PY{l+s+s1}{data/Meteo.xls}\PY{l+s+s1}{\PYZsq{}}\PY{p}{)}
\PY{n}{X} \PY{o}{=} \PY{n}{data}\PY{o}{.}\PY{n}{iloc}\PY{p}{[}\PY{p}{:}\PY{p}{,}\PY{l+m+mi}{1}\PY{p}{:}\PY{o}{\PYZhy{}}\PY{l+m+mi}{1}\PY{p}{]}
\PY{n}{Y} \PY{o}{=} \PY{n}{data}\PY{o}{.}\PY{n}{iloc}\PY{p}{[}\PY{p}{:}\PY{p}{,}\PY{o}{\PYZhy{}}\PY{l+m+mi}{1}\PY{p}{]}

\PY{n}{X2} \PY{o}{=} \PY{n}{X}\PY{o}{.}\PY{n}{replace}\PY{p}{(}\PY{n}{dic\PYZus{}X}\PY{p}{)}
\PY{n}{Y2} \PY{o}{=} \PY{n}{Y}\PY{o}{.}\PY{n}{replace}\PY{p}{(}\PY{n}{dic\PYZus{}Y}\PY{p}{)}
\end{Verbatim}
\end{tcolorbox}

    \begin{tcolorbox}[breakable, size=fbox, boxrule=1pt, pad at break*=1mm,colback=cellbackground, colframe=cellborder]
\prompt{In}{incolor}{4}{\boxspacing}
\begin{Verbatim}[commandchars=\\\{\}]
\PY{n}{data}\PY{o}{.}\PY{n}{head}\PY{p}{(}\PY{p}{)}
\end{Verbatim}
\end{tcolorbox}

            \begin{tcolorbox}[breakable, size=fbox, boxrule=.5pt, pad at break*=1mm, opacityfill=0]
\prompt{Out}{outcolor}{4}{\boxspacing}
\begin{Verbatim}[commandchars=\\\{\}]
   ID   Outlook Tempreature Humidity  Windy Class
0   1     sunny         hot     high  False     N
1   2     sunny         hot     high   True     N
2   3  overcast         hot     high  False     P
3   4      rain        mild     high  False     P
4   5      rain        cool   normal  False     P
\end{Verbatim}
\end{tcolorbox}
        
    \begin{tcolorbox}[breakable, size=fbox, boxrule=1pt, pad at break*=1mm,colback=cellbackground, colframe=cellborder]
\prompt{In}{incolor}{5}{\boxspacing}
\begin{Verbatim}[commandchars=\\\{\}]
\PY{n}{kmeans} \PY{o}{=} \PY{n}{KMeans}\PY{p}{(}\PY{n}{n\PYZus{}clusters}\PY{o}{=}\PY{l+m+mi}{2}\PY{p}{,} \PY{n}{random\PYZus{}state}\PY{o}{=}\PY{l+m+mi}{0}\PY{p}{)}\PY{o}{.}\PY{n}{fit}\PY{p}{(}\PY{n}{X2}\PY{p}{)}
\end{Verbatim}
\end{tcolorbox}

    \begin{tcolorbox}[breakable, size=fbox, boxrule=1pt, pad at break*=1mm,colback=cellbackground, colframe=cellborder]
\prompt{In}{incolor}{6}{\boxspacing}
\begin{Verbatim}[commandchars=\\\{\}]
\PY{n+nb}{print}\PY{p}{(}\PY{l+s+s2}{\PYZdq{}}\PY{l+s+s2}{Homogenity :}\PY{l+s+s2}{\PYZdq{}}\PY{p}{,} \PY{n}{metrics}\PY{o}{.}\PY{n}{homogeneity\PYZus{}score}\PY{p}{(}\PY{n}{Y2}\PY{p}{,} \PY{n}{kmeans}\PY{o}{.}\PY{n}{labels\PYZus{}}\PY{p}{)}\PY{p}{,} 
      \PY{l+s+s2}{\PYZdq{}}\PY{l+s+s2}{| Completeness :}\PY{l+s+s2}{\PYZdq{}}\PY{p}{,} \PY{n}{metrics}\PY{o}{.}\PY{n}{completeness\PYZus{}score}\PY{p}{(}\PY{n}{Y2}\PY{p}{,} \PY{n}{kmeans}\PY{o}{.}\PY{n}{labels\PYZus{}}\PY{p}{)}\PY{p}{)}
\PY{n+nb}{print}\PY{p}{(}\PY{l+s+s2}{\PYZdq{}}\PY{l+s+s2}{V\PYZhy{}measure :}\PY{l+s+s2}{\PYZdq{}}\PY{p}{,} \PY{n}{metrics}\PY{o}{.}\PY{n}{v\PYZus{}measure\PYZus{}score}\PY{p}{(}\PY{n}{Y2}\PY{p}{,} \PY{n}{kmeans}\PY{o}{.}\PY{n}{labels\PYZus{}}\PY{p}{)}\PY{p}{)}
\end{Verbatim}
\end{tcolorbox}

    \begin{Verbatim}[commandchars=\\\{\}]
Homogenity : 0.0014248244292990396 | Completeness : 0.0013598296226430605
V-measure : 0.0013915685251279558
    \end{Verbatim}

    \hypertarget{dataset-mushrooms}{%
\subsubsection{Dataset Mushrooms}\label{dataset-mushrooms}}

    \begin{tcolorbox}[breakable, size=fbox, boxrule=1pt, pad at break*=1mm,colback=cellbackground, colframe=cellborder]
\prompt{In}{incolor}{7}{\boxspacing}
\begin{Verbatim}[commandchars=\\\{\}]
\PY{n}{classes} \PY{o}{=} \PY{p}{[}\PY{l+s+s1}{\PYZsq{}}\PY{l+s+s1}{Comestible}\PY{l+s+s1}{\PYZsq{}}\PY{p}{,}\PY{l+s+s1}{\PYZsq{}}\PY{l+s+s1}{Non comestible}\PY{l+s+s1}{\PYZsq{}}\PY{p}{]}
\PY{n}{data} \PY{o}{=} \PY{n}{pd}\PY{o}{.}\PY{n}{read\PYZus{}csv}\PY{p}{(}\PY{l+s+s1}{\PYZsq{}}\PY{l+s+s1}{data/Mushroom.csv}\PY{l+s+s1}{\PYZsq{}}\PY{p}{)}
\PY{n}{X} \PY{o}{=} \PY{n}{data}\PY{o}{.}\PY{n}{iloc}\PY{p}{[}\PY{p}{:}\PY{p}{,} \PY{l+m+mi}{1}\PY{p}{:}\PY{o}{\PYZhy{}}\PY{l+m+mi}{1}\PY{p}{]}
\PY{n}{Y} \PY{o}{=} \PY{n}{data}\PY{o}{.}\PY{n}{iloc}\PY{p}{[}\PY{p}{:}\PY{p}{,}\PY{o}{\PYZhy{}}\PY{l+m+mi}{1}\PY{p}{]}
\end{Verbatim}
\end{tcolorbox}

    \begin{tcolorbox}[breakable, size=fbox, boxrule=1pt, pad at break*=1mm,colback=cellbackground, colframe=cellborder]
\prompt{In}{incolor}{8}{\boxspacing}
\begin{Verbatim}[commandchars=\\\{\}]
\PY{n}{data}\PY{o}{.}\PY{n}{head}\PY{p}{(}\PY{p}{)}
\end{Verbatim}
\end{tcolorbox}

            \begin{tcolorbox}[breakable, size=fbox, boxrule=.5pt, pad at break*=1mm, opacityfill=0]
\prompt{Out}{outcolor}{8}{\boxspacing}
\begin{Verbatim}[commandchars=\\\{\}]
   Id  Odorant  Anneau  Chapeau bombe  Piedlarge  Taches  Comestible
0   1        1       0              1          1       0           1
1   2        1       0              1          1       1           1
2   3        0       0              1          1       1           1
3   4        1       0              0          1       1           0
4   5        0       0              1          1       0           1
\end{Verbatim}
\end{tcolorbox}
        
    \begin{tcolorbox}[breakable, size=fbox, boxrule=1pt, pad at break*=1mm,colback=cellbackground, colframe=cellborder]
\prompt{In}{incolor}{9}{\boxspacing}
\begin{Verbatim}[commandchars=\\\{\}]
\PY{n}{kmeans} \PY{o}{=} \PY{n}{KMeans}\PY{p}{(}\PY{n}{n\PYZus{}clusters}\PY{o}{=}\PY{l+m+mi}{2}\PY{p}{,} \PY{n}{random\PYZus{}state}\PY{o}{=}\PY{l+m+mi}{0}\PY{p}{)}\PY{o}{.}\PY{n}{fit}\PY{p}{(}\PY{n}{X}\PY{p}{)}
\end{Verbatim}
\end{tcolorbox}

    \begin{tcolorbox}[breakable, size=fbox, boxrule=1pt, pad at break*=1mm,colback=cellbackground, colframe=cellborder]
\prompt{In}{incolor}{10}{\boxspacing}
\begin{Verbatim}[commandchars=\\\{\}]
\PY{n+nb}{print}\PY{p}{(}\PY{l+s+s2}{\PYZdq{}}\PY{l+s+s2}{Homogenity :}\PY{l+s+s2}{\PYZdq{}}\PY{p}{,} \PY{n}{metrics}\PY{o}{.}\PY{n}{homogeneity\PYZus{}score}\PY{p}{(}\PY{n}{Y}\PY{p}{,} \PY{n}{kmeans}\PY{o}{.}\PY{n}{labels\PYZus{}}\PY{p}{)}\PY{p}{,} 
      \PY{l+s+s2}{\PYZdq{}}\PY{l+s+s2}{| Completeness :}\PY{l+s+s2}{\PYZdq{}}\PY{p}{,} \PY{n}{metrics}\PY{o}{.}\PY{n}{completeness\PYZus{}score}\PY{p}{(}\PY{n}{Y}\PY{p}{,} \PY{n}{kmeans}\PY{o}{.}\PY{n}{labels\PYZus{}}\PY{p}{)}\PY{p}{)}
\PY{n+nb}{print}\PY{p}{(}\PY{l+s+s2}{\PYZdq{}}\PY{l+s+s2}{V\PYZhy{}measure :}\PY{l+s+s2}{\PYZdq{}}\PY{p}{,} \PY{n}{metrics}\PY{o}{.}\PY{n}{v\PYZus{}measure\PYZus{}score}\PY{p}{(}\PY{n}{Y}\PY{p}{,} \PY{n}{kmeans}\PY{o}{.}\PY{n}{labels\PYZus{}}\PY{p}{)}\PY{p}{)}
\end{Verbatim}
\end{tcolorbox}

    \begin{Verbatim}[commandchars=\\\{\}]
Homogenity : 0.008613264733993205 | Completeness : 0.007131315506727338
V-measure : 0.007802546326666939
    \end{Verbatim}

    \begin{tcolorbox}[breakable, size=fbox, boxrule=1pt, pad at break*=1mm,colback=cellbackground, colframe=cellborder]
\prompt{In}{incolor}{11}{\boxspacing}
\begin{Verbatim}[commandchars=\\\{\}]
\PY{n+nb}{print}\PY{p}{(}\PY{l+s+s2}{\PYZdq{}}\PY{l+s+s2}{Scores : Silhouette, Davies Bouldin}\PY{l+s+s2}{\PYZdq{}}\PY{p}{)}
\PY{n}{metrics}\PY{o}{.}\PY{n}{silhouette\PYZus{}score}\PY{p}{(}\PY{n}{X}\PY{p}{,}\PY{n}{kmeans}\PY{o}{.}\PY{n}{labels\PYZus{}}\PY{p}{)}\PY{p}{,} \PY{n}{metrics}\PY{o}{.}\PY{n}{davies\PYZus{}bouldin\PYZus{}score}\PY{p}{(}\PY{n}{X}\PY{p}{,}\PY{n}{kmeans}\PY{o}{.}\PY{n}{labels\PYZus{}}\PY{p}{)}
\end{Verbatim}
\end{tcolorbox}

    \begin{Verbatim}[commandchars=\\\{\}]
Scores : Silhouette, Davies Bouldin
    \end{Verbatim}

            \begin{tcolorbox}[breakable, size=fbox, boxrule=.5pt, pad at break*=1mm, opacityfill=0]
\prompt{Out}{outcolor}{11}{\boxspacing}
\begin{Verbatim}[commandchars=\\\{\}]
(0.29605442672133603, 1.5305580696350314)
\end{Verbatim}
\end{tcolorbox}
        
    \hypertarget{results}{%
\subsubsection{Results}\label{results}}

In the given examples, kmeans with two clusters dosen't seem to separate
well our classes in the expected groups, wether we chose to label 1 as 1
or 1 as 0. There are a lot of differences on theses basic dataframes.

    \hypertarget{explanation}{%
\subsubsection{Explanation}\label{explanation}}

Because our dataframes are small and we already know which classes we
are expecting, clustering is not an appropriate solution to get any
information. Because our data is structured, it is easier and more
accurate to use classification in order to predict our classes.

Now, let's try with another dataset, made of random numbers (unorganized
data) and try a few kmeans iterations to find out a good cluster number

    \begin{tcolorbox}[breakable, size=fbox, boxrule=1pt, pad at break*=1mm,colback=cellbackground, colframe=cellborder]
\prompt{In}{incolor}{12}{\boxspacing}
\begin{Verbatim}[commandchars=\\\{\}]
\PY{n}{data} \PY{o}{=} \PY{n}{np}\PY{o}{.}\PY{n}{random}\PY{o}{.}\PY{n}{rand}\PY{p}{(}\PY{l+m+mi}{100}\PY{p}{,}\PY{l+m+mi}{2}\PY{p}{)}    
\end{Verbatim}
\end{tcolorbox}

    \begin{tcolorbox}[breakable, size=fbox, boxrule=1pt, pad at break*=1mm,colback=cellbackground, colframe=cellborder]
\prompt{In}{incolor}{13}{\boxspacing}
\begin{Verbatim}[commandchars=\\\{\}]
\PY{n}{fig} \PY{o}{=} \PY{n}{plt}\PY{o}{.}\PY{n}{figure}\PY{p}{(}\PY{p}{)}
\PY{n}{kmeans} \PY{o}{=} \PY{n}{KMeans}\PY{p}{(}\PY{n}{n\PYZus{}clusters}\PY{o}{=}\PY{l+m+mi}{2}\PY{p}{,} \PY{n}{random\PYZus{}state}\PY{o}{=}\PY{l+m+mi}{0}\PY{p}{)}\PY{o}{.}\PY{n}{fit}\PY{p}{(}\PY{n}{data}\PY{p}{)}
\PY{n}{plt}\PY{o}{.}\PY{n}{scatter}\PY{p}{(}\PY{n}{data}\PY{p}{[}\PY{p}{:}\PY{p}{,}\PY{l+m+mi}{0}\PY{p}{]}\PY{p}{,}\PY{n}{data}\PY{p}{[}\PY{p}{:}\PY{p}{,}\PY{l+m+mi}{1}\PY{p}{]}\PY{p}{,}\PY{n}{c}\PY{o}{=}\PY{n}{kmeans}\PY{o}{.}\PY{n}{labels\PYZus{}}\PY{p}{)}
\PY{n}{plt}\PY{o}{.}\PY{n}{scatter}\PY{p}{(}\PY{n}{kmeans}\PY{o}{.}\PY{n}{cluster\PYZus{}centers\PYZus{}}\PY{p}{[}\PY{p}{:}\PY{p}{,}\PY{l+m+mi}{0}\PY{p}{]}\PY{p}{,}\PY{n}{kmeans}\PY{o}{.}\PY{n}{cluster\PYZus{}centers\PYZus{}}\PY{p}{[}\PY{p}{:}\PY{p}{,}\PY{l+m+mi}{1}\PY{p}{]}\PY{p}{,}\PY{n}{c}\PY{o}{=}\PY{l+s+s1}{\PYZsq{}}\PY{l+s+s1}{r}\PY{l+s+s1}{\PYZsq{}}\PY{p}{)}
\PY{n}{plt}\PY{o}{.}\PY{n}{show}\PY{p}{(}\PY{p}{)}
\end{Verbatim}
\end{tcolorbox}

    \begin{center}
    \adjustimage{max size={0.9\linewidth}{0.9\paperheight}}{output_39_0.png}
    \end{center}
    { \hspace*{\fill} \\}
    
    \begin{tcolorbox}[breakable, size=fbox, boxrule=1pt, pad at break*=1mm,colback=cellbackground, colframe=cellborder]
\prompt{In}{incolor}{14}{\boxspacing}
\begin{Verbatim}[commandchars=\\\{\}]
\PY{n}{fig} \PY{o}{=} \PY{n}{plt}\PY{o}{.}\PY{n}{figure}\PY{p}{(}\PY{p}{)}
\PY{n}{kmeans} \PY{o}{=} \PY{n}{KMeans}\PY{p}{(}\PY{n}{n\PYZus{}clusters}\PY{o}{=}\PY{l+m+mi}{3}\PY{p}{,} \PY{n}{random\PYZus{}state}\PY{o}{=}\PY{l+m+mi}{0}\PY{p}{)}\PY{o}{.}\PY{n}{fit}\PY{p}{(}\PY{n}{data}\PY{p}{)}
\PY{n}{plt}\PY{o}{.}\PY{n}{scatter}\PY{p}{(}\PY{n}{data}\PY{p}{[}\PY{p}{:}\PY{p}{,}\PY{l+m+mi}{0}\PY{p}{]}\PY{p}{,}\PY{n}{data}\PY{p}{[}\PY{p}{:}\PY{p}{,}\PY{l+m+mi}{1}\PY{p}{]}\PY{p}{,}\PY{n}{c}\PY{o}{=}\PY{n}{kmeans}\PY{o}{.}\PY{n}{labels\PYZus{}}\PY{p}{)}
\PY{n}{plt}\PY{o}{.}\PY{n}{scatter}\PY{p}{(}\PY{n}{kmeans}\PY{o}{.}\PY{n}{cluster\PYZus{}centers\PYZus{}}\PY{p}{[}\PY{p}{:}\PY{p}{,}\PY{l+m+mi}{0}\PY{p}{]}\PY{p}{,}\PY{n}{kmeans}\PY{o}{.}\PY{n}{cluster\PYZus{}centers\PYZus{}}\PY{p}{[}\PY{p}{:}\PY{p}{,}\PY{l+m+mi}{1}\PY{p}{]}\PY{p}{,}\PY{n}{c}\PY{o}{=}\PY{l+s+s1}{\PYZsq{}}\PY{l+s+s1}{r}\PY{l+s+s1}{\PYZsq{}}\PY{p}{)}
\PY{n}{plt}\PY{o}{.}\PY{n}{show}\PY{p}{(}\PY{p}{)}
\end{Verbatim}
\end{tcolorbox}

    \begin{center}
    \adjustimage{max size={0.9\linewidth}{0.9\paperheight}}{output_40_0.png}
    \end{center}
    { \hspace*{\fill} \\}
    
    \begin{tcolorbox}[breakable, size=fbox, boxrule=1pt, pad at break*=1mm,colback=cellbackground, colframe=cellborder]
\prompt{In}{incolor}{15}{\boxspacing}
\begin{Verbatim}[commandchars=\\\{\}]
\PY{n}{fig} \PY{o}{=} \PY{n}{plt}\PY{o}{.}\PY{n}{figure}\PY{p}{(}\PY{p}{)}
\PY{n}{kmeans} \PY{o}{=} \PY{n}{KMeans}\PY{p}{(}\PY{n}{n\PYZus{}clusters}\PY{o}{=}\PY{l+m+mi}{5}\PY{p}{,} \PY{n}{random\PYZus{}state}\PY{o}{=}\PY{l+m+mi}{0}\PY{p}{)}\PY{o}{.}\PY{n}{fit}\PY{p}{(}\PY{n}{data}\PY{p}{)}
\PY{n}{plt}\PY{o}{.}\PY{n}{scatter}\PY{p}{(}\PY{n}{data}\PY{p}{[}\PY{p}{:}\PY{p}{,}\PY{l+m+mi}{0}\PY{p}{]}\PY{p}{,}\PY{n}{data}\PY{p}{[}\PY{p}{:}\PY{p}{,}\PY{l+m+mi}{1}\PY{p}{]}\PY{p}{,}\PY{n}{c}\PY{o}{=}\PY{n}{kmeans}\PY{o}{.}\PY{n}{labels\PYZus{}}\PY{p}{)}
\PY{n}{plt}\PY{o}{.}\PY{n}{scatter}\PY{p}{(}\PY{n}{kmeans}\PY{o}{.}\PY{n}{cluster\PYZus{}centers\PYZus{}}\PY{p}{[}\PY{p}{:}\PY{p}{,}\PY{l+m+mi}{0}\PY{p}{]}\PY{p}{,}\PY{n}{kmeans}\PY{o}{.}\PY{n}{cluster\PYZus{}centers\PYZus{}}\PY{p}{[}\PY{p}{:}\PY{p}{,}\PY{l+m+mi}{1}\PY{p}{]}\PY{p}{,}\PY{n}{c}\PY{o}{=}\PY{l+s+s1}{\PYZsq{}}\PY{l+s+s1}{r}\PY{l+s+s1}{\PYZsq{}}\PY{p}{)}
\PY{n}{plt}\PY{o}{.}\PY{n}{show}\PY{p}{(}\PY{p}{)}
\end{Verbatim}
\end{tcolorbox}

    \begin{center}
    \adjustimage{max size={0.9\linewidth}{0.9\paperheight}}{output_41_0.png}
    \end{center}
    { \hspace*{\fill} \\}
    
    \begin{tcolorbox}[breakable, size=fbox, boxrule=1pt, pad at break*=1mm,colback=cellbackground, colframe=cellborder]
\prompt{In}{incolor}{16}{\boxspacing}
\begin{Verbatim}[commandchars=\\\{\}]
\PY{n}{fig} \PY{o}{=} \PY{n}{plt}\PY{o}{.}\PY{n}{figure}\PY{p}{(}\PY{p}{)}
\PY{n}{kmeans} \PY{o}{=} \PY{n}{KMeans}\PY{p}{(}\PY{n}{n\PYZus{}clusters}\PY{o}{=}\PY{l+m+mi}{10}\PY{p}{,} \PY{n}{random\PYZus{}state}\PY{o}{=}\PY{l+m+mi}{0}\PY{p}{)}\PY{o}{.}\PY{n}{fit}\PY{p}{(}\PY{n}{data}\PY{p}{)}
\PY{n}{plt}\PY{o}{.}\PY{n}{scatter}\PY{p}{(}\PY{n}{data}\PY{p}{[}\PY{p}{:}\PY{p}{,}\PY{l+m+mi}{0}\PY{p}{]}\PY{p}{,}\PY{n}{data}\PY{p}{[}\PY{p}{:}\PY{p}{,}\PY{l+m+mi}{1}\PY{p}{]}\PY{p}{,}\PY{n}{c}\PY{o}{=}\PY{n}{kmeans}\PY{o}{.}\PY{n}{labels\PYZus{}}\PY{p}{)}
\PY{n}{plt}\PY{o}{.}\PY{n}{scatter}\PY{p}{(}\PY{n}{kmeans}\PY{o}{.}\PY{n}{cluster\PYZus{}centers\PYZus{}}\PY{p}{[}\PY{p}{:}\PY{p}{,}\PY{l+m+mi}{0}\PY{p}{]}\PY{p}{,}\PY{n}{kmeans}\PY{o}{.}\PY{n}{cluster\PYZus{}centers\PYZus{}}\PY{p}{[}\PY{p}{:}\PY{p}{,}\PY{l+m+mi}{1}\PY{p}{]}\PY{p}{,}\PY{n}{c}\PY{o}{=}\PY{l+s+s1}{\PYZsq{}}\PY{l+s+s1}{r}\PY{l+s+s1}{\PYZsq{}}\PY{p}{)}
\PY{n}{plt}\PY{o}{.}\PY{n}{show}\PY{p}{(}\PY{p}{)}
\end{Verbatim}
\end{tcolorbox}

    \begin{center}
    \adjustimage{max size={0.9\linewidth}{0.9\paperheight}}{output_42_0.png}
    \end{center}
    { \hspace*{\fill} \\}
    
    Just like classification, with a big k, k-means is creating cluster
which fits the data so much a prediction wouldn't be accurate because of
the clusters being so close from each other. So how to find a good k?

We can use for example the elbow rule, which computes the distortion or
the silhouette for each k possible in a given coutinuous set

    \hypertarget{clustering-validation}{%
\subsubsection{Clustering Validation}\label{clustering-validation}}

Before trying to explore a validation technique stated above, we are
going to use the elbow method to determine an approximative k in a given
range to see which k would allow us to create great clusters.

    \begin{tcolorbox}[breakable, size=fbox, boxrule=1pt, pad at break*=1mm,colback=cellbackground, colframe=cellborder]
\prompt{In}{incolor}{17}{\boxspacing}
\begin{Verbatim}[commandchars=\\\{\}]
\PY{n}{visualizer} \PY{o}{=} \PY{n}{KElbowVisualizer}\PY{p}{(}\PY{n}{kmeans}\PY{p}{,} \PY{n}{k}\PY{o}{=}\PY{p}{(}\PY{l+m+mi}{2}\PY{p}{,}\PY{l+m+mi}{20}\PY{p}{)}\PY{p}{,} \PY{n}{metric}\PY{o}{=}\PY{l+s+s1}{\PYZsq{}}\PY{l+s+s1}{silhouette}\PY{l+s+s1}{\PYZsq{}}\PY{p}{)}
\PY{n}{visualizer}\PY{o}{.}\PY{n}{fit}\PY{p}{(}\PY{n}{data}\PY{p}{)}
\end{Verbatim}
\end{tcolorbox}

    \begin{Verbatim}[commandchars=\\\{\}]
C:\textbackslash{}Users\textbackslash{}CAJug\textbackslash{}anaconda3\textbackslash{}envs\textbackslash{}R\_py37\textbackslash{}lib\textbackslash{}site-packages\textbackslash{}sklearn\textbackslash{}base.py:213:
FutureWarning: From version 0.24, get\_params will raise an AttributeError if a
parameter cannot be retrieved as an instance attribute. Previously it would
return None.
  FutureWarning)
    \end{Verbatim}

            \begin{tcolorbox}[breakable, size=fbox, boxrule=.5pt, pad at break*=1mm, opacityfill=0]
\prompt{Out}{outcolor}{17}{\boxspacing}
\begin{Verbatim}[commandchars=\\\{\}]
KElbowVisualizer(ax=<AxesSubplot:>, k=None, metric='silhouette', model=None)
\end{Verbatim}
\end{tcolorbox}
        
    \begin{center}
    \adjustimage{max size={0.9\linewidth}{0.9\paperheight}}{output_45_2.png}
    \end{center}
    { \hspace*{\fill} \\}
    
    \begin{tcolorbox}[breakable, size=fbox, boxrule=1pt, pad at break*=1mm,colback=cellbackground, colframe=cellborder]
\prompt{In}{incolor}{18}{\boxspacing}
\begin{Verbatim}[commandchars=\\\{\}]
\PY{n}{visualizer} \PY{o}{=} \PY{n}{KElbowVisualizer}\PY{p}{(}\PY{n}{kmeans}\PY{p}{,} \PY{n}{k}\PY{o}{=}\PY{p}{(}\PY{l+m+mi}{2}\PY{p}{,}\PY{l+m+mi}{20}\PY{p}{)}\PY{p}{,}\PY{n}{metric}\PY{o}{=}\PY{l+s+s1}{\PYZsq{}}\PY{l+s+s1}{distortion}\PY{l+s+s1}{\PYZsq{}}\PY{p}{)}
\PY{n}{visualizer}\PY{o}{.}\PY{n}{fit}\PY{p}{(}\PY{n}{data}\PY{p}{)}
\end{Verbatim}
\end{tcolorbox}

    \begin{Verbatim}[commandchars=\\\{\}]
C:\textbackslash{}Users\textbackslash{}CAJug\textbackslash{}anaconda3\textbackslash{}envs\textbackslash{}R\_py37\textbackslash{}lib\textbackslash{}site-packages\textbackslash{}sklearn\textbackslash{}base.py:213:
FutureWarning: From version 0.24, get\_params will raise an AttributeError if a
parameter cannot be retrieved as an instance attribute. Previously it would
return None.
  FutureWarning)
    \end{Verbatim}

            \begin{tcolorbox}[breakable, size=fbox, boxrule=.5pt, pad at break*=1mm, opacityfill=0]
\prompt{Out}{outcolor}{18}{\boxspacing}
\begin{Verbatim}[commandchars=\\\{\}]
KElbowVisualizer(ax=<AxesSubplot:>, k=None, model=None)
\end{Verbatim}
\end{tcolorbox}
        
    \begin{center}
    \adjustimage{max size={0.9\linewidth}{0.9\paperheight}}{output_46_2.png}
    \end{center}
    { \hspace*{\fill} \\}
    
    On this generation, the best k according to the silhouette score is k =
10. For this k we notice a peak on the distortion score which then leads
to a smoother slope. So there is a chance we are getting a good number
of clusters for this distribution.

    \hypertarget{bootstrapping}{%
\subsubsection{Bootstrapping}\label{bootstrapping}}

    \begin{tcolorbox}[breakable, size=fbox, boxrule=1pt, pad at break*=1mm,colback=cellbackground, colframe=cellborder]
\prompt{In}{incolor}{19}{\boxspacing}
\begin{Verbatim}[commandchars=\\\{\}]
\PY{k}{def} \PY{n+nf}{bootstrapping}\PY{p}{(}\PY{n}{data}\PY{o}{=}\PY{n}{np}\PY{o}{.}\PY{n}{random}\PY{o}{.}\PY{n}{rand}\PY{p}{(}\PY{l+m+mi}{100}\PY{p}{,}\PY{l+m+mi}{2}\PY{p}{)}\PY{p}{,} \PY{n}{iteration}\PY{o}{=}\PY{l+m+mi}{10}\PY{p}{,} \PY{n}{k}\PY{o}{=}\PY{l+m+mi}{5}\PY{p}{)}\PY{p}{:}
    \PY{c+c1}{\PYZsh{} Variables to work with}
    \PY{n}{size} \PY{o}{=} \PY{n}{data}\PY{o}{.}\PY{n}{shape}\PY{p}{[}\PY{l+m+mi}{0}\PY{p}{]}
    \PY{n}{data\PYZus{}df} \PY{o}{=} \PY{n}{pd}\PY{o}{.}\PY{n}{DataFrame}\PY{p}{(}\PY{n}{data}\PY{p}{)}
    \PY{n}{df} \PY{o}{=} \PY{n}{pd}\PY{o}{.}\PY{n}{DataFrame}\PY{p}{(}\PY{p}{)}
    \PY{n}{samples} \PY{o}{=} \PY{p}{[}\PY{p}{]}
    \PY{n}{inertia} \PY{o}{=} \PY{p}{[}\PY{p}{]}
    \PY{n}{silhouette} \PY{o}{=} \PY{p}{[}\PY{p}{]}

    \PY{k}{for} \PY{n}{j} \PY{o+ow}{in} \PY{n+nb}{range}\PY{p}{(}\PY{n}{iteration}\PY{p}{)}\PY{p}{:}
        \PY{c+c1}{\PYZsh{} Build samples}
        \PY{n}{i} \PY{o}{=} \PY{l+m+mi}{0}
        \PY{k}{while} \PY{n}{i} \PY{o}{\PYZlt{}} \PY{n}{size} \PY{p}{:}
            \PY{n}{df} \PY{o}{=} \PY{n}{df}\PY{o}{.}\PY{n}{append}\PY{p}{(}\PY{n}{data\PYZus{}df}\PY{o}{.}\PY{n}{sample}\PY{p}{(}\PY{p}{)}\PY{p}{)}
            \PY{n}{i}\PY{o}{+}\PY{o}{=}\PY{l+m+mi}{1}
        \PY{n}{samples}\PY{o}{.}\PY{n}{append}\PY{p}{(}\PY{n}{df}\PY{p}{)}
  
    \PY{k}{for} \PY{n}{sample} \PY{o+ow}{in} \PY{n}{samples}\PY{p}{:}
        \PY{n}{km} \PY{o}{=} \PY{n}{KMeans}\PY{p}{(}\PY{n}{n\PYZus{}clusters}\PY{o}{=}\PY{n}{k}\PY{p}{,} \PY{n}{random\PYZus{}state}\PY{o}{=}\PY{l+m+mi}{0}\PY{p}{)}\PY{o}{.}\PY{n}{fit}\PY{p}{(}\PY{n}{sample}\PY{p}{)}
        \PY{n}{inertia}\PY{o}{.}\PY{n}{append}\PY{p}{(}\PY{n}{km}\PY{o}{.}\PY{n}{inertia\PYZus{}}\PY{p}{)}
    
        \PY{n}{silhouette}\PY{o}{.}\PY{n}{append}\PY{p}{(}\PY{n}{metrics}\PY{o}{.}\PY{n}{silhouette\PYZus{}score}\PY{p}{(}\PY{n}{sample}\PY{p}{,}\PY{n}{km}\PY{o}{.}\PY{n}{labels\PYZus{}}\PY{p}{)}\PY{p}{)}
  
    \PY{n+nb}{print}\PY{p}{(}\PY{l+s+s2}{\PYZdq{}}\PY{l+s+s2}{Average Silhouette score =}\PY{l+s+s2}{\PYZdq{}}\PY{p}{,} \PY{n}{np}\PY{o}{.}\PY{n}{mean}\PY{p}{(}\PY{n}{silhouette}\PY{p}{)}\PY{p}{)}
    \PY{n+nb}{print}\PY{p}{(}\PY{l+s+s2}{\PYZdq{}}\PY{l+s+s2}{Average inertia =}\PY{l+s+s2}{\PYZdq{}}\PY{p}{,}\PY{n}{np}\PY{o}{.}\PY{n}{mean}\PY{p}{(}\PY{n}{inertia}\PY{p}{)}\PY{p}{)}
\end{Verbatim}
\end{tcolorbox}

    \begin{tcolorbox}[breakable, size=fbox, boxrule=1pt, pad at break*=1mm,colback=cellbackground, colframe=cellborder]
\prompt{In}{incolor}{21}{\boxspacing}
\begin{Verbatim}[commandchars=\\\{\}]
\PY{c+c1}{\PYZsh{} Iteration : numbers of sample generated and fit to the model, k : a given k}
\PY{n}{bootstrapping}\PY{p}{(}\PY{n}{data}\PY{p}{,}\PY{n}{iteration}\PY{o}{=}\PY{l+m+mi}{20}\PY{p}{,}\PY{n}{k}\PY{o}{=}\PY{l+m+mi}{6}\PY{p}{)}
\end{Verbatim}
\end{tcolorbox}

    \begin{Verbatim}[commandchars=\\\{\}]
Average Silhouette score = 0.45809107224876505
Average inertia = 24.182625389992833
    \end{Verbatim}

    Bootstrapping shows us our kmeans over dimensional coordinates works
well since its mean value is almost at 0.5 in a {[}-1, 1{]} scope. The
inertia is also high considering distances of individuals to their
centers is below sqrt(2) in this dimensional space, meaning the clusters
are coherent.

    \hypertarget{r-part}{%
\subsection{R Part}\label{r-part}}

    The dataset : https://www.kaggle.com/carlolepelaars/toy-dataset

Content :

Note: All data is fictional. The data has been generated so that their
distributions are convenient for statistical analysis.

\begin{itemize}
\tightlist
\item
  Number: A simple index number for each row
\item
  City: The location of a person (Dallas, New York City, Los Angeles,
  Mountain View, Boston, Washington D.C., San Diego and Austin)
\item
  Gender: Gender of a person (Male or Female)
\item
  Age: The age of a person (Ranging from 25 to 65 years)
\item
  Income: Annual income of a person (Ranging from -674 to 177175)
\item
  Illness: Is the person Ill? (Yes or No)
\end{itemize}

    \hypertarget{dataset-analysis-and-preprocessing}{%
\subsubsection{Dataset Analysis And
preprocessing}\label{dataset-analysis-and-preprocessing}}

    \hypertarget{imports}{%
\paragraph{Imports}\label{imports}}

    \begin{tcolorbox}[breakable, size=fbox, boxrule=1pt, pad at break*=1mm,colback=cellbackground, colframe=cellborder]
\prompt{In}{incolor}{2}{\boxspacing}
\begin{Verbatim}[commandchars=\\\{\}]
\PY{n}{library}\PY{p}{(}\PY{n}{namespace}\PY{p}{)}
\PY{n}{registerNamespace}\PY{p}{(}\PY{l+s+s1}{\PYZsq{}}\PY{l+s+s1}{psy}\PY{l+s+s1}{\PYZsq{}}\PY{p}{,} \PY{n}{loadNamespace}\PY{p}{(}\PY{l+s+s1}{\PYZsq{}}\PY{l+s+s1}{psych}\PY{l+s+s1}{\PYZsq{}}\PY{p}{)}\PY{p}{)}
\PY{n}{library}\PY{p}{(}\PY{n}{ggplot2}\PY{p}{)}
\PY{n}{library}\PY{p}{(}\PY{n}{reshape2}\PY{p}{)}
\PY{n}{library}\PY{p}{(}\PY{l+s+s2}{\PYZdq{}}\PY{l+s+s2}{IRdisplay}\PY{l+s+s2}{\PYZdq{}}\PY{p}{)}
\PY{n}{library}\PY{p}{(}\PY{n}{ade4}\PY{p}{)}
\PY{n}{library}\PY{p}{(}\PY{n}{factoextra}\PY{p}{)}
\PY{n}{library}\PY{p}{(}\PY{n}{NbClust}\PY{p}{)}
\PY{n}{library}\PY{p}{(}\PY{n}{bootcluster}\PY{p}{)}
\end{Verbatim}
\end{tcolorbox}

    
    \begin{Verbatim}[commandchars=\\\{\}]
<environment: namespace:psych>
    \end{Verbatim}

    
    \begin{Verbatim}[commandchars=\\\{\}]
Warning message:
"package 'ggplot2' was built under R version 3.6.3"Warning message:
"package 'factoextra' was built under R version 3.6.3"Welcome! Want to learn
more? See two factoextra-related books at https://goo.gl/ve3WBa
Warning message:
"package 'bootcluster' was built under R version 3.6.3"Registered S3 method
overwritten by 'GGally':
  method from
  +.gg   ggplot2
Registered S3 method overwritten by 'sets':
  method        from
  print.element ggplot2
    \end{Verbatim}

    \begin{tcolorbox}[breakable, size=fbox, boxrule=1pt, pad at break*=1mm,colback=cellbackground, colframe=cellborder]
\prompt{In}{incolor}{3}{\boxspacing}
\begin{Verbatim}[commandchars=\\\{\}]
\PY{n}{csv} \PY{o}{\PYZlt{}}\PY{o}{\PYZhy{}} \PY{n}{read}\PY{o}{.}\PY{n}{csv}\PY{p}{(}\PY{l+s+s2}{\PYZdq{}}\PY{l+s+s2}{data/toy\PYZus{}dataset.csv}\PY{l+s+s2}{\PYZdq{}}\PY{p}{,} \PY{n}{header} \PY{o}{=} \PY{n}{TRUE}\PY{p}{)}
\PY{n}{head}\PY{p}{(}\PY{n}{csv}\PY{p}{[}\PY{p}{,}\PY{l+m+mi}{2}\PY{p}{:}\PY{n}{ncol}\PY{p}{(}\PY{n}{csv}\PY{p}{)}\PY{p}{]}\PY{p}{)}
\end{Verbatim}
\end{tcolorbox}

    \begin{tabular}{r|lllll}
 City & Gender & Age & Income & Illness\\
\hline
	 Dallas & Male   & 41     & 40367  & No    \\
	 Dallas & Male   & 54     & 45084  & No    \\
	 Dallas & Male   & 42     & 52483  & No    \\
	 Dallas & Male   & 40     & 40941  & No    \\
	 Dallas & Male   & 46     & 50289  & No    \\
	 Dallas & Female & 36     & 50786  & No    \\
\end{tabular}


    
    \hypertarget{data-understanding---univariate-analysis}{%
\paragraph{Data Understanding - Univariate
Analysis}\label{data-understanding---univariate-analysis}}

    \begin{tcolorbox}[breakable, size=fbox, boxrule=1pt, pad at break*=1mm,colback=cellbackground, colframe=cellborder]
\prompt{In}{incolor}{4}{\boxspacing}
\begin{Verbatim}[commandchars=\\\{\}]
\PY{n}{df} \PY{o}{\PYZlt{}}\PY{o}{\PYZhy{}} \PY{k}{as}\PY{o}{.}\PY{n}{data}\PY{o}{.}\PY{n}{frame}\PY{p}{(}\PY{n}{table}\PY{p}{(}\PY{n}{csv}\PY{p}{[}\PY{p}{,}\PY{l+m+mi}{2}\PY{p}{]}\PY{p}{)}\PY{p}{)}
\PY{n}{df}\PY{p}{[}\PY{n}{order}\PY{p}{(}\PY{n}{df}\PY{err}{\PYZdl{}}\PY{n}{Freq}\PY{p}{)}\PY{p}{,}\PY{p}{]}
\end{Verbatim}
\end{tcolorbox}

    \begin{tabular}{r|ll}
  & Var1 & Freq\\
\hline
	7 & San Diego       &  4881          \\
	8 & Washington D.C. &  8120          \\
	2 & Boston          &  8301          \\
	1 & Austin          & 12292          \\
	5 & Mountain View   & 14219          \\
	3 & Dallas          & 19707          \\
	4 & Los Angeles     & 32173          \\
	6 & New York City   & 50307          \\
\end{tabular}


    
    We can see that the cities are unequally proportionnate in this dataset,
and this could lead to bias in our models.

    \begin{tcolorbox}[breakable, size=fbox, boxrule=1pt, pad at break*=1mm,colback=cellbackground, colframe=cellborder]
\prompt{In}{incolor}{5}{\boxspacing}
\begin{Verbatim}[commandchars=\\\{\}]
\PY{n}{df} \PY{o}{\PYZlt{}}\PY{o}{\PYZhy{}} \PY{k}{as}\PY{o}{.}\PY{n}{data}\PY{o}{.}\PY{n}{frame}\PY{p}{(}\PY{n}{table}\PY{p}{(}\PY{n}{csv}\PY{p}{[}\PY{p}{,}\PY{l+m+mi}{3}\PY{p}{]}\PY{p}{)}\PY{p}{)}
\PY{n}{df}\PY{p}{[}\PY{n}{order}\PY{p}{(}\PY{n}{df}\PY{err}{\PYZdl{}}\PY{n}{Freq}\PY{p}{)}\PY{p}{,}\PY{p}{]}
\end{Verbatim}
\end{tcolorbox}

    \begin{tabular}{r|ll}
 Var1 & Freq\\
\hline
	 Female & 66200 \\
	 Male   & 83800 \\
\end{tabular}


    
    Women are less present in this dataset than men, we should take it in
account too

    \begin{tcolorbox}[breakable, size=fbox, boxrule=1pt, pad at break*=1mm,colback=cellbackground, colframe=cellborder]
\prompt{In}{incolor}{6}{\boxspacing}
\begin{Verbatim}[commandchars=\\\{\}]
\PY{n}{hist}\PY{p}{(}\PY{n}{csv}\PY{p}{[}\PY{p}{,}\PY{l+m+mi}{4}\PY{p}{]}\PY{p}{,} \PY{n}{breaks}\PY{o}{=}\PY{n}{pretty}\PY{p}{(}\PY{n}{csv}\PY{p}{[}\PY{p}{,}\PY{l+m+mi}{4}\PY{p}{]}\PY{p}{,}\PY{l+m+mi}{5}\PY{p}{)}\PY{p}{,} \PY{n}{col}\PY{o}{=}\PY{l+s+s2}{\PYZdq{}}\PY{l+s+s2}{red}\PY{l+s+s2}{\PYZdq{}}\PY{p}{,} \PY{n}{xlab}\PY{o}{=}\PY{l+s+s2}{\PYZdq{}}\PY{l+s+s2}{Age categories}\PY{l+s+s2}{\PYZdq{}}\PY{p}{,} \PY{n}{main}\PY{o}{=}\PY{l+s+s2}{\PYZdq{}}\PY{l+s+s2}{Distribution of ages}\PY{l+s+s2}{\PYZdq{}}\PY{p}{)}
\end{Verbatim}
\end{tcolorbox}

    \begin{center}
    \adjustimage{max size={0.9\linewidth}{0.9\paperheight}}{output_63_0.png}
    \end{center}
    { \hspace*{\fill} \\}
    
    Same conclusion as before, we can see that the ages follows a normal
distribution, centered around 45 years old. Then, old (over 60 years
old) and young (below 20 years old) people are less represented.

    \begin{tcolorbox}[breakable, size=fbox, boxrule=1pt, pad at break*=1mm,colback=cellbackground, colframe=cellborder]
\prompt{In}{incolor}{7}{\boxspacing}
\begin{Verbatim}[commandchars=\\\{\}]
\PY{n}{hist}\PY{p}{(}\PY{n}{csv}\PY{p}{[}\PY{p}{,}\PY{l+m+mi}{5}\PY{p}{]}\PY{p}{,} \PY{n}{breaks}\PY{o}{=}\PY{n}{pretty}\PY{p}{(}\PY{n}{csv}\PY{p}{[}\PY{p}{,}\PY{l+m+mi}{5}\PY{p}{]}\PY{p}{,}\PY{l+m+mi}{1000}\PY{p}{)}\PY{p}{,} \PY{n}{col}\PY{o}{=}\PY{l+s+s2}{\PYZdq{}}\PY{l+s+s2}{red}\PY{l+s+s2}{\PYZdq{}}\PY{p}{,} \PY{n}{xlab}\PY{o}{=}\PY{l+s+s2}{\PYZdq{}}\PY{l+s+s2}{Incomes Category}\PY{l+s+s2}{\PYZdq{}}\PY{p}{,} \PY{n}{main}\PY{o}{=}\PY{l+s+s2}{\PYZdq{}}\PY{l+s+s2}{Distribution of incomes}\PY{l+s+s2}{\PYZdq{}}\PY{p}{)}
\end{Verbatim}
\end{tcolorbox}

    \begin{center}
    \adjustimage{max size={0.9\linewidth}{0.9\paperheight}}{output_65_0.png}
    \end{center}
    { \hspace*{\fill} \\}
    
    Here is a relatively more complex distribution with a high peak around
95000 units per year, and with some less high peaks : fortunate people
around 130000 units per year, and less fortunate people around 50000
units per year.

    \begin{tcolorbox}[breakable, size=fbox, boxrule=1pt, pad at break*=1mm,colback=cellbackground, colframe=cellborder]
\prompt{In}{incolor}{8}{\boxspacing}
\begin{Verbatim}[commandchars=\\\{\}]
\PY{n}{df} \PY{o}{\PYZlt{}}\PY{o}{\PYZhy{}} \PY{k}{as}\PY{o}{.}\PY{n}{data}\PY{o}{.}\PY{n}{frame}\PY{p}{(}\PY{n}{table}\PY{p}{(}\PY{n}{csv}\PY{p}{[}\PY{p}{,}\PY{l+m+mi}{6}\PY{p}{]}\PY{p}{)}\PY{p}{)}
\PY{n}{df}\PY{p}{[}\PY{n}{order}\PY{p}{(}\PY{n}{df}\PY{err}{\PYZdl{}}\PY{n}{Freq}\PY{p}{)}\PY{p}{,}\PY{p}{]}
\end{Verbatim}
\end{tcolorbox}

    \begin{tabular}{r|ll}
  & Var1 & Freq\\
\hline
	2 & Yes    &  12139\\
	1 & No     & 137861\\
\end{tabular}


    
    Lastly, we can see that our dataset contains less than a 10\% ratio on
Ill and Safe people.

    \hypertarget{data-understanding---bivariate-analysis}{%
\paragraph{Data Understanding - Bivariate
Analysis}\label{data-understanding---bivariate-analysis}}

    For the simplicity of this analysis, we will only try to evaluate the
combination of each row with Illness.

    \begin{tcolorbox}[breakable, size=fbox, boxrule=1pt, pad at break*=1mm,colback=cellbackground, colframe=cellborder]
\prompt{In}{incolor}{9}{\boxspacing}
\begin{Verbatim}[commandchars=\\\{\}]
\PY{n}{df} \PY{o}{\PYZlt{}}\PY{o}{\PYZhy{}} \PY{k}{as}\PY{o}{.}\PY{n}{data}\PY{o}{.}\PY{n}{frame}\PY{p}{(}\PY{n}{table}\PY{p}{(}\PY{n}{csv}\PY{p}{[}\PY{p}{,}\PY{n}{c}\PY{p}{(}\PY{l+m+mi}{2}\PY{p}{,}\PY{l+m+mi}{6}\PY{p}{)}\PY{p}{]}\PY{p}{)}\PY{p}{)}
\PY{n}{df}\PY{p}{[}\PY{n}{order}\PY{p}{(}\PY{n}{df}\PY{err}{\PYZdl{}}\PY{n}{City}\PY{p}{)}\PY{p}{,}\PY{p}{]}
\end{Verbatim}
\end{tcolorbox}

    \begin{tabular}{r|lll}
  & City & Illness & Freq\\
\hline
	1 & Austin          & No              & 11281          \\
	9 & Austin          & Yes             &  1011          \\
	2 & Boston          & No              &  7615          \\
	10 & Boston          & Yes             &   686          \\
	3 & Dallas          & No              & 18094          \\
	11 & Dallas          & Yes             &  1613          \\
	4 & Los Angeles     & No              & 29605          \\
	12 & Los Angeles     & Yes             &  2568          \\
	5 & Mountain View   & No              & 13041          \\
	13 & Mountain View   & Yes             &  1178          \\
	6 & New York City   & No              & 46286          \\
	14 & New York City   & Yes             &  4021          \\
	7 & San Diego       & No              &  4487          \\
	15 & San Diego       & Yes             &   394          \\
	8 & Washington D.C. & No              &  7452          \\
	16 & Washington D.C. & Yes             &   668          \\
\end{tabular}


    
    \begin{tcolorbox}[breakable, size=fbox, boxrule=1pt, pad at break*=1mm,colback=cellbackground, colframe=cellborder]
\prompt{In}{incolor}{10}{\boxspacing}
\begin{Verbatim}[commandchars=\\\{\}]
\PY{n}{prop} \PY{o}{\PYZlt{}}\PY{o}{\PYZhy{}} \PY{n}{c}\PY{p}{(}\PY{p}{)}
\PY{n}{rows} \PY{o}{=} \PY{n}{nrow}\PY{p}{(}\PY{n}{df}\PY{p}{)}\PY{o}{/}\PY{l+m+mi}{2}
\PY{k}{for}\PY{p}{(}\PY{n}{i} \PY{o+ow}{in} \PY{l+m+mi}{1}\PY{p}{:}\PY{n}{rows}\PY{p}{)}\PY{p}{\PYZob{}}
    \PY{n}{prop} \PY{o}{\PYZlt{}}\PY{o}{\PYZhy{}} \PY{n}{c}\PY{p}{(}\PY{n}{prop}\PY{p}{,} \PY{n}{df}\PY{err}{\PYZdl{}}\PY{n}{Freq}\PY{p}{[}\PY{n}{i}\PY{o}{+}\PY{n}{rows}\PY{p}{]}\PY{o}{/}\PY{n}{df}\PY{err}{\PYZdl{}}\PY{n}{Freq}\PY{p}{[}\PY{n}{i}\PY{p}{]}\PY{p}{)}
\PY{p}{\PYZcb{}}
\PY{n}{prop} \PY{o}{\PYZlt{}}\PY{o}{\PYZhy{}} \PY{k}{as}\PY{o}{.}\PY{n}{data}\PY{o}{.}\PY{n}{frame}\PY{p}{(}\PY{n+nb}{list}\PY{p}{(}\PY{l+s+s2}{\PYZdq{}}\PY{l+s+s2}{City}\PY{l+s+s2}{\PYZdq{}}\PY{o}{=}\PY{n}{df}\PY{err}{\PYZdl{}}\PY{n}{City}\PY{p}{,} \PY{l+s+s2}{\PYZdq{}}\PY{l+s+s2}{Proportion}\PY{l+s+s2}{\PYZdq{}}\PY{o}{=}\PY{n}{prop}\PY{p}{)}\PY{p}{)}
\end{Verbatim}
\end{tcolorbox}

    \begin{tcolorbox}[breakable, size=fbox, boxrule=1pt, pad at break*=1mm,colback=cellbackground, colframe=cellborder]
\prompt{In}{incolor}{11}{\boxspacing}
\begin{Verbatim}[commandchars=\\\{\}]
\PY{n}{ggplot}\PY{p}{(}\PY{n}{data}\PY{o}{=}\PY{n}{prop}\PY{p}{,} \PY{n}{aes}\PY{p}{(}\PY{n}{x}\PY{o}{=}\PY{n}{City}\PY{p}{,} \PY{n}{y}\PY{o}{=}\PY{n}{Proportion}\PY{p}{)}\PY{p}{)} \PY{o}{+}
  \PY{n}{geom\PYZus{}bar}\PY{p}{(}\PY{n}{stat}\PY{o}{=}\PY{l+s+s2}{\PYZdq{}}\PY{l+s+s2}{identity}\PY{l+s+s2}{\PYZdq{}}\PY{p}{,} \PY{n}{fill}\PY{o}{=}\PY{l+s+s2}{\PYZdq{}}\PY{l+s+s2}{steelblue}\PY{l+s+s2}{\PYZdq{}}\PY{p}{)} \PY{o}{+}
  \PY{n}{geom\PYZus{}text}\PY{p}{(}\PY{n}{aes}\PY{p}{(}\PY{n}{label}\PY{o}{=}\PY{n+nb}{round}\PY{p}{(}\PY{n}{Proportion}\PY{p}{,}\PY{l+m+mi}{4}\PY{p}{)}\PY{p}{)}\PY{p}{,} \PY{n}{vjust}\PY{o}{=}\PY{l+m+mf}{1.6}\PY{p}{,} \PY{n}{color}\PY{o}{=}\PY{l+s+s2}{\PYZdq{}}\PY{l+s+s2}{white}\PY{l+s+s2}{\PYZdq{}}\PY{p}{,} \PY{n}{size}\PY{o}{=}\PY{l+m+mi}{3}\PY{p}{)} \PY{o}{+}
  \PY{n}{theme}\PY{p}{(}\PY{n}{axis}\PY{o}{.}\PY{n}{text}\PY{o}{.}\PY{n}{y}\PY{o}{=}\PY{n}{element\PYZus{}blank}\PY{p}{(}\PY{p}{)}\PY{p}{,}
        \PY{n}{axis}\PY{o}{.}\PY{n}{ticks}\PY{o}{.}\PY{n}{y}\PY{o}{=}\PY{n}{element\PYZus{}blank}\PY{p}{(}\PY{p}{)}\PY{p}{)}
\end{Verbatim}
\end{tcolorbox}

    \begin{center}
    \adjustimage{max size={0.9\linewidth}{0.9\paperheight}}{output_73_0.png}
    \end{center}
    { \hspace*{\fill} \\}
    
    As we can see, each city as quite the same amount of ill person, it
means that it is not a deterministic factor : there are no real
correlation between those columns.

    \begin{tcolorbox}[breakable, size=fbox, boxrule=1pt, pad at break*=1mm,colback=cellbackground, colframe=cellborder]
\prompt{In}{incolor}{12}{\boxspacing}
\begin{Verbatim}[commandchars=\\\{\}]
\PY{n}{df} \PY{o}{\PYZlt{}}\PY{o}{\PYZhy{}} \PY{k}{as}\PY{o}{.}\PY{n}{data}\PY{o}{.}\PY{n}{frame}\PY{p}{(}\PY{n}{table}\PY{p}{(}\PY{n}{csv}\PY{p}{[}\PY{p}{,}\PY{n}{c}\PY{p}{(}\PY{l+m+mi}{3}\PY{p}{,}\PY{l+m+mi}{6}\PY{p}{)}\PY{p}{]}\PY{p}{)}\PY{p}{)}
\PY{n}{df}\PY{p}{[}\PY{n}{order}\PY{p}{(}\PY{n}{df}\PY{err}{\PYZdl{}}\PY{n}{Gender}\PY{p}{)}\PY{p}{,}\PY{p}{]}
\end{Verbatim}
\end{tcolorbox}

    \begin{tabular}{r|lll}
  & Gender & Illness & Freq\\
\hline
	1 & Female & No     & 60869 \\
	3 & Female & Yes    &  5331 \\
	2 & Male   & No     & 76992 \\
	4 & Male   & Yes    &  6808 \\
\end{tabular}


    
    \begin{tcolorbox}[breakable, size=fbox, boxrule=1pt, pad at break*=1mm,colback=cellbackground, colframe=cellborder]
\prompt{In}{incolor}{13}{\boxspacing}
\begin{Verbatim}[commandchars=\\\{\}]
\PY{n}{prop} \PY{o}{\PYZlt{}}\PY{o}{\PYZhy{}} \PY{n}{c}\PY{p}{(}\PY{p}{)}
\PY{n}{rows} \PY{o}{=} \PY{n}{nrow}\PY{p}{(}\PY{n}{df}\PY{p}{)}\PY{o}{/}\PY{l+m+mi}{2}
\PY{k}{for}\PY{p}{(}\PY{n}{i} \PY{o+ow}{in} \PY{l+m+mi}{1}\PY{p}{:}\PY{n}{rows}\PY{p}{)}\PY{p}{\PYZob{}}
    \PY{n}{prop} \PY{o}{\PYZlt{}}\PY{o}{\PYZhy{}} \PY{n}{c}\PY{p}{(}\PY{n}{prop}\PY{p}{,} \PY{n}{df}\PY{err}{\PYZdl{}}\PY{n}{Freq}\PY{p}{[}\PY{n}{i}\PY{o}{+}\PY{n}{rows}\PY{p}{]}\PY{o}{/}\PY{n}{df}\PY{err}{\PYZdl{}}\PY{n}{Freq}\PY{p}{[}\PY{n}{i}\PY{p}{]}\PY{p}{)}
\PY{p}{\PYZcb{}}
\PY{n}{prop} \PY{o}{\PYZlt{}}\PY{o}{\PYZhy{}} \PY{k}{as}\PY{o}{.}\PY{n}{data}\PY{o}{.}\PY{n}{frame}\PY{p}{(}\PY{n+nb}{list}\PY{p}{(}\PY{l+s+s2}{\PYZdq{}}\PY{l+s+s2}{Gender}\PY{l+s+s2}{\PYZdq{}}\PY{o}{=}\PY{n}{df}\PY{err}{\PYZdl{}}\PY{n}{Gender}\PY{p}{,} \PY{l+s+s2}{\PYZdq{}}\PY{l+s+s2}{Proportion}\PY{l+s+s2}{\PYZdq{}}\PY{o}{=}\PY{n}{prop}\PY{p}{)}\PY{p}{)}
\end{Verbatim}
\end{tcolorbox}

    \begin{tcolorbox}[breakable, size=fbox, boxrule=1pt, pad at break*=1mm,colback=cellbackground, colframe=cellborder]
\prompt{In}{incolor}{14}{\boxspacing}
\begin{Verbatim}[commandchars=\\\{\}]
\PY{n}{ggplot}\PY{p}{(}\PY{n}{data}\PY{o}{=}\PY{n}{prop}\PY{p}{,} \PY{n}{aes}\PY{p}{(}\PY{n}{x}\PY{o}{=}\PY{n}{Gender}\PY{p}{,} \PY{n}{y}\PY{o}{=}\PY{n}{Proportion}\PY{p}{)}\PY{p}{)} \PY{o}{+}
  \PY{n}{geom\PYZus{}bar}\PY{p}{(}\PY{n}{stat}\PY{o}{=}\PY{l+s+s2}{\PYZdq{}}\PY{l+s+s2}{identity}\PY{l+s+s2}{\PYZdq{}}\PY{p}{,} \PY{n}{fill}\PY{o}{=}\PY{l+s+s2}{\PYZdq{}}\PY{l+s+s2}{steelblue}\PY{l+s+s2}{\PYZdq{}}\PY{p}{)} \PY{o}{+}
  \PY{n}{geom\PYZus{}text}\PY{p}{(}\PY{n}{aes}\PY{p}{(}\PY{n}{label}\PY{o}{=}\PY{n+nb}{round}\PY{p}{(}\PY{n}{Proportion}\PY{p}{,}\PY{l+m+mi}{4}\PY{p}{)}\PY{p}{)}\PY{p}{,} \PY{n}{vjust}\PY{o}{=}\PY{l+m+mf}{1.6}\PY{p}{,} \PY{n}{color}\PY{o}{=}\PY{l+s+s2}{\PYZdq{}}\PY{l+s+s2}{white}\PY{l+s+s2}{\PYZdq{}}\PY{p}{,} \PY{n}{size}\PY{o}{=}\PY{l+m+mi}{3}\PY{p}{)} \PY{o}{+}
  \PY{n}{theme}\PY{p}{(}\PY{n}{axis}\PY{o}{.}\PY{n}{text}\PY{o}{.}\PY{n}{y}\PY{o}{=}\PY{n}{element\PYZus{}blank}\PY{p}{(}\PY{p}{)}\PY{p}{,}
        \PY{n}{axis}\PY{o}{.}\PY{n}{ticks}\PY{o}{.}\PY{n}{y}\PY{o}{=}\PY{n}{element\PYZus{}blank}\PY{p}{(}\PY{p}{)}\PY{p}{)}
\end{Verbatim}
\end{tcolorbox}

    \begin{center}
    \adjustimage{max size={0.9\linewidth}{0.9\paperheight}}{output_77_0.png}
    \end{center}
    { \hspace*{\fill} \\}
    
    Same conclusion for the gender, both have the same ratio.

\begin{center}\rule{0.5\linewidth}{0.5pt}\end{center}

Let's now have a look at ages by creating categories.

    \begin{tcolorbox}[breakable, size=fbox, boxrule=1pt, pad at break*=1mm,colback=cellbackground, colframe=cellborder]
\prompt{In}{incolor}{15}{\boxspacing}
\begin{Verbatim}[commandchars=\\\{\}]
\PY{n}{age\PYZus{}categories} \PY{o}{=} \PY{n+nb}{list}\PY{p}{(}\PY{l+s+s2}{\PYZdq{}}\PY{l+s+s2}{Ages}\PY{l+s+s2}{\PYZdq{}}\PY{o}{=}\PY{n+nb}{list}\PY{p}{(}\PY{l+s+s2}{\PYZdq{}}\PY{l+s+s2}{Min}\PY{l+s+s2}{\PYZdq{}}\PY{o}{=}\PY{n}{c}\PY{p}{(}\PY{l+m+mi}{20}\PY{p}{,}\PY{l+m+mi}{30}\PY{p}{,}\PY{l+m+mi}{40}\PY{p}{,}\PY{l+m+mi}{50}\PY{p}{,}\PY{l+m+mi}{60}\PY{p}{)}\PY{p}{,} \PY{l+s+s2}{\PYZdq{}}\PY{l+s+s2}{Max}\PY{l+s+s2}{\PYZdq{}}\PY{o}{=}\PY{n}{c}\PY{p}{(}\PY{l+m+mi}{29}\PY{p}{,}\PY{l+m+mi}{39}\PY{p}{,}\PY{l+m+mi}{49}\PY{p}{,}\PY{l+m+mi}{59}\PY{p}{,}\PY{l+m+mi}{150}\PY{p}{)}\PY{p}{)}\PY{p}{,} \PY{l+s+s2}{\PYZdq{}}\PY{l+s+s2}{Ill}\PY{l+s+s2}{\PYZdq{}}\PY{o}{=}\PY{n}{c}\PY{p}{(}\PY{p}{)}\PY{p}{,} \PY{l+s+s2}{\PYZdq{}}\PY{l+s+s2}{Safe}\PY{l+s+s2}{\PYZdq{}}\PY{o}{=}\PY{n}{c}\PY{p}{(}\PY{p}{)}\PY{p}{)}
\PY{k}{for}\PY{p}{(}\PY{n}{i} \PY{o+ow}{in} \PY{l+m+mi}{1}\PY{p}{:}\PY{n}{length}\PY{p}{(}\PY{n}{age\PYZus{}categories}\PY{err}{\PYZdl{}}\PY{n}{Ages}\PY{err}{\PYZdl{}}\PY{n}{Min}\PY{p}{)}\PY{p}{)}\PY{p}{\PYZob{}}
    \PY{n}{age\PYZus{}categories}\PY{err}{\PYZdl{}}\PY{n}{Ill} \PY{o}{\PYZlt{}}\PY{o}{\PYZhy{}} \PY{n}{c}\PY{p}{(}\PY{n}{age\PYZus{}categories}\PY{err}{\PYZdl{}}\PY{n}{Ill}\PY{p}{,}\PY{l+m+mi}{0}\PY{p}{)}
    \PY{n}{age\PYZus{}categories}\PY{err}{\PYZdl{}}\PY{n}{Safe} \PY{o}{\PYZlt{}}\PY{o}{\PYZhy{}} \PY{n}{c}\PY{p}{(}\PY{n}{age\PYZus{}categories}\PY{err}{\PYZdl{}}\PY{n}{Safe}\PY{p}{,}\PY{l+m+mi}{0}\PY{p}{)}
\PY{p}{\PYZcb{}}
\PY{k}{for}\PY{p}{(}\PY{n}{i} \PY{o+ow}{in} \PY{l+m+mi}{1}\PY{p}{:}\PY{n}{nrow}\PY{p}{(}\PY{n}{csv}\PY{p}{)}\PY{p}{)}\PY{p}{\PYZob{}}
    \PY{k}{for}\PY{p}{(}\PY{n}{j} \PY{o+ow}{in} \PY{l+m+mi}{1}\PY{p}{:}\PY{n}{length}\PY{p}{(}\PY{n}{age\PYZus{}categories}\PY{err}{\PYZdl{}}\PY{n}{Ages}\PY{err}{\PYZdl{}}\PY{n}{Min}\PY{p}{)}\PY{p}{)}\PY{p}{\PYZob{}}
        \PY{k}{if}\PY{p}{(}\PY{n}{csv}\PY{err}{\PYZdl{}}\PY{n}{Age}\PY{p}{[}\PY{n}{i}\PY{p}{]} \PY{o}{\PYZlt{}}\PY{o}{=} \PY{n}{age\PYZus{}categories}\PY{err}{\PYZdl{}}\PY{n}{Ages}\PY{err}{\PYZdl{}}\PY{n}{Max}\PY{p}{[}\PY{n}{j}\PY{p}{]} \PY{o}{\PYZam{}}\PY{o}{\PYZam{}} \PY{n}{csv}\PY{err}{\PYZdl{}}\PY{n}{Age}\PY{p}{[}\PY{n}{i}\PY{p}{]} \PY{o}{\PYZgt{}}\PY{o}{=} \PY{n}{age\PYZus{}categories}\PY{err}{\PYZdl{}}\PY{n}{Ages}\PY{err}{\PYZdl{}}\PY{n}{Min}\PY{p}{[}\PY{n}{j}\PY{p}{]}\PY{p}{)}\PY{p}{\PYZob{}}
            \PY{k}{if}\PY{p}{(}\PY{n}{csv}\PY{err}{\PYZdl{}}\PY{n}{Illness}\PY{p}{[}\PY{n}{i}\PY{p}{]} \PY{o}{==} \PY{l+s+s2}{\PYZdq{}}\PY{l+s+s2}{No}\PY{l+s+s2}{\PYZdq{}}\PY{p}{)}\PY{p}{\PYZob{}}
                \PY{n}{age\PYZus{}categories}\PY{err}{\PYZdl{}}\PY{n}{Safe}\PY{p}{[}\PY{n}{j}\PY{p}{]} \PY{o}{=} \PY{n}{age\PYZus{}categories}\PY{err}{\PYZdl{}}\PY{n}{Safe}\PY{p}{[}\PY{n}{j}\PY{p}{]} \PY{o}{+} \PY{l+m+mi}{1}
            \PY{p}{\PYZcb{}}\PY{k}{else}\PY{p}{\PYZob{}}
                \PY{n}{age\PYZus{}categories}\PY{err}{\PYZdl{}}\PY{n}{Ill}\PY{p}{[}\PY{n}{j}\PY{p}{]} \PY{o}{=} \PY{n}{age\PYZus{}categories}\PY{err}{\PYZdl{}}\PY{n}{Ill}\PY{p}{[}\PY{n}{j}\PY{p}{]} \PY{o}{+} \PY{l+m+mi}{1}
            \PY{p}{\PYZcb{}}
        \PY{p}{\PYZcb{}}
    \PY{p}{\PYZcb{}}
\PY{p}{\PYZcb{}}
\PY{n}{prop} \PY{o}{\PYZlt{}}\PY{o}{\PYZhy{}} \PY{n+nb}{list}\PY{p}{(}\PY{l+s+s2}{\PYZdq{}}\PY{l+s+s2}{Age}\PY{l+s+s2}{\PYZdq{}}\PY{o}{=}\PY{n}{c}\PY{p}{(}\PY{p}{)}\PY{p}{,} \PY{l+s+s2}{\PYZdq{}}\PY{l+s+s2}{Proportion}\PY{l+s+s2}{\PYZdq{}}\PY{o}{=}\PY{n}{c}\PY{p}{(}\PY{p}{)}\PY{p}{)}
\PY{k}{for}\PY{p}{(}\PY{n}{i} \PY{o+ow}{in} \PY{l+m+mi}{1}\PY{p}{:}\PY{n}{length}\PY{p}{(}\PY{n}{age\PYZus{}categories}\PY{err}{\PYZdl{}}\PY{n}{Ages}\PY{err}{\PYZdl{}}\PY{n}{Min}\PY{p}{)}\PY{p}{)}\PY{p}{\PYZob{}}
    \PY{n}{prop}\PY{err}{\PYZdl{}}\PY{n}{Age} \PY{o}{\PYZlt{}}\PY{o}{\PYZhy{}} \PY{n}{c}\PY{p}{(}\PY{n}{prop}\PY{err}{\PYZdl{}}\PY{n}{Age}\PY{p}{,} \PY{n}{paste0}\PY{p}{(}\PY{n}{age\PYZus{}categories}\PY{err}{\PYZdl{}}\PY{n}{Ages}\PY{err}{\PYZdl{}}\PY{n}{Min}\PY{p}{[}\PY{n}{i}\PY{p}{]}\PY{p}{,}\PY{l+s+s2}{\PYZdq{}}\PY{l+s+s2}{\PYZhy{}}\PY{l+s+s2}{\PYZdq{}}\PY{p}{,}\PY{n}{age\PYZus{}categories}\PY{err}{\PYZdl{}}\PY{n}{Ages}\PY{err}{\PYZdl{}}\PY{n}{Max}\PY{p}{[}\PY{n}{i}\PY{p}{]}\PY{p}{)}\PY{p}{)}
    \PY{n}{prop}\PY{err}{\PYZdl{}}\PY{n}{Proportion} \PY{o}{\PYZlt{}}\PY{o}{\PYZhy{}} \PY{n}{c}\PY{p}{(}\PY{n}{prop}\PY{err}{\PYZdl{}}\PY{n}{Proportion}\PY{p}{,} \PY{n}{age\PYZus{}categories}\PY{err}{\PYZdl{}}\PY{n}{Ill}\PY{p}{[}\PY{n}{i}\PY{p}{]}\PY{o}{/}\PY{n}{age\PYZus{}categories}\PY{err}{\PYZdl{}}\PY{n}{Safe}\PY{p}{[}\PY{n}{i}\PY{p}{]}\PY{p}{)}
\PY{p}{\PYZcb{}}
\PY{n}{prop} \PY{o}{\PYZlt{}}\PY{o}{\PYZhy{}} \PY{k}{as}\PY{o}{.}\PY{n}{data}\PY{o}{.}\PY{n}{frame}\PY{p}{(}\PY{n}{prop}\PY{p}{)}
\end{Verbatim}
\end{tcolorbox}

    \begin{tcolorbox}[breakable, size=fbox, boxrule=1pt, pad at break*=1mm,colback=cellbackground, colframe=cellborder]
\prompt{In}{incolor}{16}{\boxspacing}
\begin{Verbatim}[commandchars=\\\{\}]
\PY{n}{prop}
\end{Verbatim}
\end{tcolorbox}

    \begin{tabular}{r|ll}
 Age & Proportion\\
\hline
	 20-29      & 0.08796707\\
	 30-39      & 0.08626458\\
	 40-49      & 0.08928675\\
	 50-59      & 0.08741544\\
	 60-150     & 0.09028147\\
\end{tabular}


    
    \begin{tcolorbox}[breakable, size=fbox, boxrule=1pt, pad at break*=1mm,colback=cellbackground, colframe=cellborder]
\prompt{In}{incolor}{17}{\boxspacing}
\begin{Verbatim}[commandchars=\\\{\}]
\PY{n}{ggplot}\PY{p}{(}\PY{n}{data}\PY{o}{=}\PY{n}{prop}\PY{p}{,} \PY{n}{aes}\PY{p}{(}\PY{n}{x}\PY{o}{=}\PY{n}{Age}\PY{p}{,} \PY{n}{y}\PY{o}{=}\PY{n}{Proportion}\PY{p}{)}\PY{p}{)} \PY{o}{+}
  \PY{n}{geom\PYZus{}bar}\PY{p}{(}\PY{n}{stat}\PY{o}{=}\PY{l+s+s2}{\PYZdq{}}\PY{l+s+s2}{identity}\PY{l+s+s2}{\PYZdq{}}\PY{p}{,} \PY{n}{fill}\PY{o}{=}\PY{l+s+s2}{\PYZdq{}}\PY{l+s+s2}{steelblue}\PY{l+s+s2}{\PYZdq{}}\PY{p}{)} \PY{o}{+}
  \PY{n}{geom\PYZus{}text}\PY{p}{(}\PY{n}{aes}\PY{p}{(}\PY{n}{label}\PY{o}{=}\PY{n+nb}{round}\PY{p}{(}\PY{n}{Proportion}\PY{p}{,}\PY{l+m+mi}{4}\PY{p}{)}\PY{p}{)}\PY{p}{,} \PY{n}{vjust}\PY{o}{=}\PY{l+m+mf}{1.6}\PY{p}{,} \PY{n}{color}\PY{o}{=}\PY{l+s+s2}{\PYZdq{}}\PY{l+s+s2}{white}\PY{l+s+s2}{\PYZdq{}}\PY{p}{,} \PY{n}{size}\PY{o}{=}\PY{l+m+mi}{3}\PY{p}{)} \PY{o}{+}
  \PY{n}{theme}\PY{p}{(}\PY{n}{axis}\PY{o}{.}\PY{n}{text}\PY{o}{.}\PY{n}{y}\PY{o}{=}\PY{n}{element\PYZus{}blank}\PY{p}{(}\PY{p}{)}\PY{p}{,}
        \PY{n}{axis}\PY{o}{.}\PY{n}{ticks}\PY{o}{.}\PY{n}{y}\PY{o}{=}\PY{n}{element\PYZus{}blank}\PY{p}{(}\PY{p}{)}\PY{p}{)}
\end{Verbatim}
\end{tcolorbox}

    \begin{center}
    \adjustimage{max size={0.9\linewidth}{0.9\paperheight}}{output_81_0.png}
    \end{center}
    { \hspace*{\fill} \\}
    
    We can see that the ratio is quite the same for each category of age
(which is a bit strange considering reality). It is probably due to the
fact that those datas are not real and were uniformly generated.

    \begin{tcolorbox}[breakable, size=fbox, boxrule=1pt, pad at break*=1mm,colback=cellbackground, colframe=cellborder]
\prompt{In}{incolor}{18}{\boxspacing}
\begin{Verbatim}[commandchars=\\\{\}]
\PY{n}{income\PYZus{}categories} \PY{o}{=} \PY{n+nb}{list}\PY{p}{(}\PY{l+s+s2}{\PYZdq{}}\PY{l+s+s2}{Incomes}\PY{l+s+s2}{\PYZdq{}}\PY{o}{=}\PY{n+nb}{list}\PY{p}{(}\PY{l+s+s2}{\PYZdq{}}\PY{l+s+s2}{Min}\PY{l+s+s2}{\PYZdq{}}\PY{o}{=}\PY{n}{c}\PY{p}{(}\PY{l+m+mi}{0}\PY{p}{,}\PY{l+m+mi}{60001}\PY{p}{,}\PY{l+m+mi}{125001}\PY{p}{)}\PY{p}{,} \PY{l+s+s2}{\PYZdq{}}\PY{l+s+s2}{Max}\PY{l+s+s2}{\PYZdq{}}\PY{o}{=}\PY{n}{c}\PY{p}{(}\PY{l+m+mi}{60000}\PY{p}{,}\PY{l+m+mi}{125000}\PY{p}{,}\PY{l+m+mi}{200000}\PY{p}{)}\PY{p}{)}\PY{p}{,} \PY{l+s+s2}{\PYZdq{}}\PY{l+s+s2}{Ill}\PY{l+s+s2}{\PYZdq{}}\PY{o}{=}\PY{n}{c}\PY{p}{(}\PY{p}{)}\PY{p}{,} \PY{l+s+s2}{\PYZdq{}}\PY{l+s+s2}{Safe}\PY{l+s+s2}{\PYZdq{}}\PY{o}{=}\PY{n}{c}\PY{p}{(}\PY{p}{)}\PY{p}{)}
\PY{k}{for}\PY{p}{(}\PY{n}{i} \PY{o+ow}{in} \PY{l+m+mi}{1}\PY{p}{:}\PY{n}{length}\PY{p}{(}\PY{n}{income\PYZus{}categories}\PY{err}{\PYZdl{}}\PY{n}{Incomes}\PY{err}{\PYZdl{}}\PY{n}{Min}\PY{p}{)}\PY{p}{)}\PY{p}{\PYZob{}}
    \PY{n}{income\PYZus{}categories}\PY{err}{\PYZdl{}}\PY{n}{Ill} \PY{o}{\PYZlt{}}\PY{o}{\PYZhy{}} \PY{n}{c}\PY{p}{(}\PY{n}{income\PYZus{}categories}\PY{err}{\PYZdl{}}\PY{n}{Ill}\PY{p}{,}\PY{l+m+mi}{0}\PY{p}{)}
    \PY{n}{income\PYZus{}categories}\PY{err}{\PYZdl{}}\PY{n}{Safe} \PY{o}{\PYZlt{}}\PY{o}{\PYZhy{}} \PY{n}{c}\PY{p}{(}\PY{n}{income\PYZus{}categories}\PY{err}{\PYZdl{}}\PY{n}{Safe}\PY{p}{,}\PY{l+m+mi}{0}\PY{p}{)}
\PY{p}{\PYZcb{}}
\PY{k}{for}\PY{p}{(}\PY{n}{i} \PY{o+ow}{in} \PY{l+m+mi}{1}\PY{p}{:}\PY{n}{nrow}\PY{p}{(}\PY{n}{csv}\PY{p}{)}\PY{p}{)}\PY{p}{\PYZob{}}
    \PY{k}{for}\PY{p}{(}\PY{n}{j} \PY{o+ow}{in} \PY{l+m+mi}{1}\PY{p}{:}\PY{n}{length}\PY{p}{(}\PY{n}{income\PYZus{}categories}\PY{err}{\PYZdl{}}\PY{n}{Incomes}\PY{err}{\PYZdl{}}\PY{n}{Min}\PY{p}{)}\PY{p}{)}\PY{p}{\PYZob{}}
        \PY{k}{if}\PY{p}{(}\PY{n}{csv}\PY{err}{\PYZdl{}}\PY{n}{Income}\PY{p}{[}\PY{n}{i}\PY{p}{]} \PY{o}{\PYZlt{}}\PY{o}{=} \PY{n}{income\PYZus{}categories}\PY{err}{\PYZdl{}}\PY{n}{Incomes}\PY{err}{\PYZdl{}}\PY{n}{Max}\PY{p}{[}\PY{n}{j}\PY{p}{]} \PY{o}{\PYZam{}}\PY{o}{\PYZam{}} \PY{n}{csv}\PY{err}{\PYZdl{}}\PY{n}{Income}\PY{p}{[}\PY{n}{i}\PY{p}{]} \PY{o}{\PYZgt{}}\PY{o}{=} \PY{n}{income\PYZus{}categories}\PY{err}{\PYZdl{}}\PY{n}{Incomes}\PY{err}{\PYZdl{}}\PY{n}{Min}\PY{p}{[}\PY{n}{j}\PY{p}{]}\PY{p}{)}\PY{p}{\PYZob{}}
            \PY{k}{if}\PY{p}{(}\PY{n}{csv}\PY{err}{\PYZdl{}}\PY{n}{Illness}\PY{p}{[}\PY{n}{i}\PY{p}{]} \PY{o}{==} \PY{l+s+s2}{\PYZdq{}}\PY{l+s+s2}{No}\PY{l+s+s2}{\PYZdq{}}\PY{p}{)}\PY{p}{\PYZob{}}
                \PY{n}{income\PYZus{}categories}\PY{err}{\PYZdl{}}\PY{n}{Safe}\PY{p}{[}\PY{n}{j}\PY{p}{]} \PY{o}{=} \PY{n}{income\PYZus{}categories}\PY{err}{\PYZdl{}}\PY{n}{Safe}\PY{p}{[}\PY{n}{j}\PY{p}{]} \PY{o}{+} \PY{l+m+mi}{1}
            \PY{p}{\PYZcb{}}\PY{k}{else}\PY{p}{\PYZob{}}
                \PY{n}{income\PYZus{}categories}\PY{err}{\PYZdl{}}\PY{n}{Ill}\PY{p}{[}\PY{n}{j}\PY{p}{]} \PY{o}{=} \PY{n}{income\PYZus{}categories}\PY{err}{\PYZdl{}}\PY{n}{Ill}\PY{p}{[}\PY{n}{j}\PY{p}{]} \PY{o}{+} \PY{l+m+mi}{1}
            \PY{p}{\PYZcb{}}
        \PY{p}{\PYZcb{}}
    \PY{p}{\PYZcb{}}
\PY{p}{\PYZcb{}}
\PY{n}{prop} \PY{o}{\PYZlt{}}\PY{o}{\PYZhy{}} \PY{n+nb}{list}\PY{p}{(}\PY{l+s+s2}{\PYZdq{}}\PY{l+s+s2}{Incomes}\PY{l+s+s2}{\PYZdq{}}\PY{o}{=}\PY{n}{c}\PY{p}{(}\PY{p}{)}\PY{p}{,} \PY{l+s+s2}{\PYZdq{}}\PY{l+s+s2}{Proportion}\PY{l+s+s2}{\PYZdq{}}\PY{o}{=}\PY{n}{c}\PY{p}{(}\PY{p}{)}\PY{p}{)}
\PY{k}{for}\PY{p}{(}\PY{n}{i} \PY{o+ow}{in} \PY{l+m+mi}{1}\PY{p}{:}\PY{n}{length}\PY{p}{(}\PY{n}{income\PYZus{}categories}\PY{err}{\PYZdl{}}\PY{n}{Incomes}\PY{err}{\PYZdl{}}\PY{n}{Min}\PY{p}{)}\PY{p}{)}\PY{p}{\PYZob{}}
    \PY{n}{prop}\PY{err}{\PYZdl{}}\PY{n}{Incomes} \PY{o}{\PYZlt{}}\PY{o}{\PYZhy{}} \PY{n}{c}\PY{p}{(}\PY{n}{prop}\PY{err}{\PYZdl{}}\PY{n}{Incomes}\PY{p}{,} \PY{n}{paste0}\PY{p}{(}\PY{n}{income\PYZus{}categories}\PY{err}{\PYZdl{}}\PY{n}{Incomes}\PY{err}{\PYZdl{}}\PY{n}{Min}\PY{p}{[}\PY{n}{i}\PY{p}{]}\PY{p}{,}\PY{l+s+s2}{\PYZdq{}}\PY{l+s+s2}{\PYZhy{}}\PY{l+s+s2}{\PYZdq{}}\PY{p}{,}\PY{n}{income\PYZus{}categories}\PY{err}{\PYZdl{}}\PY{n}{Incomes}\PY{err}{\PYZdl{}}\PY{n}{Max}\PY{p}{[}\PY{n}{i}\PY{p}{]}\PY{p}{)}\PY{p}{)}
    \PY{n}{prop}\PY{err}{\PYZdl{}}\PY{n}{Proportion} \PY{o}{\PYZlt{}}\PY{o}{\PYZhy{}} \PY{n}{c}\PY{p}{(}\PY{n}{prop}\PY{err}{\PYZdl{}}\PY{n}{Proportion}\PY{p}{,} \PY{n}{income\PYZus{}categories}\PY{err}{\PYZdl{}}\PY{n}{Ill}\PY{p}{[}\PY{n}{i}\PY{p}{]}\PY{o}{/}\PY{n}{income\PYZus{}categories}\PY{err}{\PYZdl{}}\PY{n}{Safe}\PY{p}{[}\PY{n}{i}\PY{p}{]}\PY{p}{)}
\PY{p}{\PYZcb{}}
\PY{n}{prop} \PY{o}{\PYZlt{}}\PY{o}{\PYZhy{}} \PY{k}{as}\PY{o}{.}\PY{n}{data}\PY{o}{.}\PY{n}{frame}\PY{p}{(}\PY{n}{prop}\PY{p}{)}
\end{Verbatim}
\end{tcolorbox}

    \begin{tcolorbox}[breakable, size=fbox, boxrule=1pt, pad at break*=1mm,colback=cellbackground, colframe=cellborder]
\prompt{In}{incolor}{19}{\boxspacing}
\begin{Verbatim}[commandchars=\\\{\}]
\PY{n}{prop}
\end{Verbatim}
\end{tcolorbox}

    \begin{tabular}{r|ll}
 Incomes & Proportion\\
\hline
	 0-60000      & 0.08848909  \\
	 60001-125000 & 0.08770788  \\
	 125001-2e+05 & 0.09067758  \\
\end{tabular}


    
    \begin{tcolorbox}[breakable, size=fbox, boxrule=1pt, pad at break*=1mm,colback=cellbackground, colframe=cellborder]
\prompt{In}{incolor}{20}{\boxspacing}
\begin{Verbatim}[commandchars=\\\{\}]
\PY{n}{ggplot}\PY{p}{(}\PY{n}{data}\PY{o}{=}\PY{n}{prop}\PY{p}{,} \PY{n}{aes}\PY{p}{(}\PY{n}{x}\PY{o}{=}\PY{n}{Incomes}\PY{p}{,} \PY{n}{y}\PY{o}{=}\PY{n}{Proportion}\PY{p}{)}\PY{p}{)} \PY{o}{+}
  \PY{n}{geom\PYZus{}bar}\PY{p}{(}\PY{n}{stat}\PY{o}{=}\PY{l+s+s2}{\PYZdq{}}\PY{l+s+s2}{identity}\PY{l+s+s2}{\PYZdq{}}\PY{p}{,} \PY{n}{fill}\PY{o}{=}\PY{l+s+s2}{\PYZdq{}}\PY{l+s+s2}{steelblue}\PY{l+s+s2}{\PYZdq{}}\PY{p}{)} \PY{o}{+}
  \PY{n}{geom\PYZus{}text}\PY{p}{(}\PY{n}{aes}\PY{p}{(}\PY{n}{label}\PY{o}{=}\PY{n+nb}{round}\PY{p}{(}\PY{n}{Proportion}\PY{p}{,}\PY{l+m+mi}{4}\PY{p}{)}\PY{p}{)}\PY{p}{,} \PY{n}{vjust}\PY{o}{=}\PY{l+m+mf}{1.6}\PY{p}{,} \PY{n}{color}\PY{o}{=}\PY{l+s+s2}{\PYZdq{}}\PY{l+s+s2}{white}\PY{l+s+s2}{\PYZdq{}}\PY{p}{,} \PY{n}{size}\PY{o}{=}\PY{l+m+mi}{3}\PY{p}{)} \PY{o}{+}
  \PY{n}{theme}\PY{p}{(}\PY{n}{axis}\PY{o}{.}\PY{n}{text}\PY{o}{.}\PY{n}{y}\PY{o}{=}\PY{n}{element\PYZus{}blank}\PY{p}{(}\PY{p}{)}\PY{p}{,}
        \PY{n}{axis}\PY{o}{.}\PY{n}{ticks}\PY{o}{.}\PY{n}{y}\PY{o}{=}\PY{n}{element\PYZus{}blank}\PY{p}{(}\PY{p}{)}\PY{p}{)}
\end{Verbatim}
\end{tcolorbox}

    \begin{center}
    \adjustimage{max size={0.9\linewidth}{0.9\paperheight}}{output_85_0.png}
    \end{center}
    { \hspace*{\fill} \\}
    
    Same conclusion as before, we can't say that any column in our dataset
can be linearly correlated to Illness.

    \hypertarget{preprocessing-data---extract-binary-categorical-variable}{%
\paragraph{Preprocessing Data - Extract Binary Categorical
Variable}\label{preprocessing-data---extract-binary-categorical-variable}}

    \begin{tcolorbox}[breakable, size=fbox, boxrule=1pt, pad at break*=1mm,colback=cellbackground, colframe=cellborder]
\prompt{In}{incolor}{21}{\boxspacing}
\begin{Verbatim}[commandchars=\\\{\}]
\PY{n}{df\PYZus{}g} \PY{o}{\PYZlt{}}\PY{o}{\PYZhy{}} \PY{k}{as}\PY{o}{.}\PY{n}{data}\PY{o}{.}\PY{n}{frame}\PY{p}{(}\PY{n}{table}\PY{p}{(}\PY{n}{csv}\PY{p}{[}\PY{p}{,}\PY{l+m+mi}{3}\PY{p}{]}\PY{p}{)}\PY{p}{)}
\PY{n}{df\PYZus{}i} \PY{o}{\PYZlt{}}\PY{o}{\PYZhy{}} \PY{k}{as}\PY{o}{.}\PY{n}{data}\PY{o}{.}\PY{n}{frame}\PY{p}{(}\PY{n}{table}\PY{p}{(}\PY{n}{csv}\PY{p}{[}\PY{p}{,}\PY{l+m+mi}{6}\PY{p}{]}\PY{p}{)}\PY{p}{)}
\PY{n}{rows} \PY{o}{=} \PY{n}{nrow}\PY{p}{(}\PY{n}{csv}\PY{p}{)}
\PY{n}{gender} \PY{o}{\PYZlt{}}\PY{o}{\PYZhy{}} \PY{n}{integer}\PY{p}{(}\PY{n}{rows}\PY{p}{)}
\PY{n}{illness} \PY{o}{\PYZlt{}}\PY{o}{\PYZhy{}} \PY{n}{integer}\PY{p}{(}\PY{n}{rows}\PY{p}{)}
\PY{k}{for}\PY{p}{(}\PY{n}{i} \PY{o+ow}{in} \PY{l+m+mi}{1}\PY{p}{:}\PY{n}{rows}\PY{p}{)}\PY{p}{\PYZob{}}
    \PY{k}{for}\PY{p}{(}\PY{n}{j} \PY{o+ow}{in} \PY{l+m+mi}{1}\PY{p}{:}\PY{n}{nrow}\PY{p}{(}\PY{n}{df\PYZus{}g}\PY{p}{)}\PY{p}{)}\PY{p}{\PYZob{}}
        \PY{k}{if}\PY{p}{(}\PY{n}{csv}\PY{err}{\PYZdl{}}\PY{n}{Gender}\PY{p}{[}\PY{n}{i}\PY{p}{]} \PY{o}{==} \PY{n}{df\PYZus{}g}\PY{err}{\PYZdl{}}\PY{n}{Var1}\PY{p}{[}\PY{n}{j}\PY{p}{]}\PY{p}{)}\PY{p}{\PYZob{}}
            \PY{n}{gender}\PY{p}{[}\PY{n}{i}\PY{p}{]} \PY{o}{\PYZlt{}}\PY{o}{\PYZhy{}} \PY{n}{j}\PY{o}{\PYZhy{}}\PY{l+m+mi}{1}
            \PY{k}{break}
        \PY{p}{\PYZcb{}}
    \PY{p}{\PYZcb{}}
    \PY{k}{for}\PY{p}{(}\PY{n}{j} \PY{o+ow}{in} \PY{l+m+mi}{1}\PY{p}{:}\PY{n}{nrow}\PY{p}{(}\PY{n}{df\PYZus{}i}\PY{p}{)}\PY{p}{)}\PY{p}{\PYZob{}}
        \PY{k}{if}\PY{p}{(}\PY{n}{csv}\PY{err}{\PYZdl{}}\PY{n}{Illness}\PY{p}{[}\PY{n}{i}\PY{p}{]} \PY{o}{==} \PY{n}{df\PYZus{}i}\PY{err}{\PYZdl{}}\PY{n}{Var1}\PY{p}{[}\PY{n}{j}\PY{p}{]}\PY{p}{)}\PY{p}{\PYZob{}}
            \PY{n}{illness}\PY{p}{[}\PY{n}{i}\PY{p}{]} \PY{o}{\PYZlt{}}\PY{o}{\PYZhy{}} \PY{n}{j}\PY{o}{\PYZhy{}}\PY{l+m+mi}{1}
            \PY{k}{break}
        \PY{p}{\PYZcb{}}
    \PY{p}{\PYZcb{}}
\PY{p}{\PYZcb{}}
\PY{n}{gender\PYZus{}label} \PY{o}{\PYZlt{}}\PY{o}{\PYZhy{}} \PY{n}{df\PYZus{}g}\PY{err}{\PYZdl{}}\PY{n}{Var1}
\PY{n}{illness\PYZus{}label} \PY{o}{\PYZlt{}}\PY{o}{\PYZhy{}} \PY{n}{df\PYZus{}i}\PY{err}{\PYZdl{}}\PY{n}{Var1}
\end{Verbatim}
\end{tcolorbox}

    \hypertarget{preprocessing-data---extract-multi-categorical-variable-ordinals}{%
\paragraph{Preprocessing Data - Extract Multi Categorical Variable
(ordinals)}\label{preprocessing-data---extract-multi-categorical-variable-ordinals}}

    \begin{tcolorbox}[breakable, size=fbox, boxrule=1pt, pad at break*=1mm,colback=cellbackground, colframe=cellborder]
\prompt{In}{incolor}{22}{\boxspacing}
\begin{Verbatim}[commandchars=\\\{\}]
\PY{n}{age\PYZus{}categories} \PY{o}{=} \PY{n+nb}{list}\PY{p}{(}\PY{l+s+s2}{\PYZdq{}}\PY{l+s+s2}{Ages}\PY{l+s+s2}{\PYZdq{}}\PY{o}{=}\PY{n+nb}{list}\PY{p}{(}\PY{l+s+s2}{\PYZdq{}}\PY{l+s+s2}{Min}\PY{l+s+s2}{\PYZdq{}}\PY{o}{=}\PY{n}{c}\PY{p}{(}\PY{l+m+mi}{20}\PY{p}{,}\PY{l+m+mi}{30}\PY{p}{,}\PY{l+m+mi}{40}\PY{p}{,}\PY{l+m+mi}{50}\PY{p}{,}\PY{l+m+mi}{60}\PY{p}{)}\PY{p}{,} \PY{l+s+s2}{\PYZdq{}}\PY{l+s+s2}{Max}\PY{l+s+s2}{\PYZdq{}}\PY{o}{=}\PY{n}{c}\PY{p}{(}\PY{l+m+mi}{29}\PY{p}{,}\PY{l+m+mi}{39}\PY{p}{,}\PY{l+m+mi}{49}\PY{p}{,}\PY{l+m+mi}{59}\PY{p}{,}\PY{l+m+mi}{150}\PY{p}{)}\PY{p}{)}\PY{p}{)}
\PY{n}{age} \PY{o}{\PYZlt{}}\PY{o}{\PYZhy{}} \PY{n}{integer}\PY{p}{(}\PY{n}{rows}\PY{p}{)}
\PY{k}{for}\PY{p}{(}\PY{n}{i} \PY{o+ow}{in} \PY{l+m+mi}{1}\PY{p}{:}\PY{n}{nrow}\PY{p}{(}\PY{n}{csv}\PY{p}{)}\PY{p}{)}\PY{p}{\PYZob{}}
    \PY{k}{for}\PY{p}{(}\PY{n}{j} \PY{o+ow}{in} \PY{l+m+mi}{1}\PY{p}{:}\PY{n}{length}\PY{p}{(}\PY{n}{age\PYZus{}categories}\PY{err}{\PYZdl{}}\PY{n}{Ages}\PY{err}{\PYZdl{}}\PY{n}{Min}\PY{p}{)}\PY{p}{)}\PY{p}{\PYZob{}}
        \PY{k}{if}\PY{p}{(}\PY{n}{csv}\PY{err}{\PYZdl{}}\PY{n}{Age}\PY{p}{[}\PY{n}{i}\PY{p}{]} \PY{o}{\PYZlt{}}\PY{o}{=} \PY{n}{age\PYZus{}categories}\PY{err}{\PYZdl{}}\PY{n}{Ages}\PY{err}{\PYZdl{}}\PY{n}{Max}\PY{p}{[}\PY{n}{j}\PY{p}{]} \PY{o}{\PYZam{}}\PY{o}{\PYZam{}} \PY{n}{csv}\PY{err}{\PYZdl{}}\PY{n}{Age}\PY{p}{[}\PY{n}{i}\PY{p}{]} \PY{o}{\PYZgt{}}\PY{o}{=} \PY{n}{age\PYZus{}categories}\PY{err}{\PYZdl{}}\PY{n}{Ages}\PY{err}{\PYZdl{}}\PY{n}{Min}\PY{p}{[}\PY{n}{j}\PY{p}{]}\PY{p}{)}\PY{p}{\PYZob{}}
            \PY{n}{age}\PY{p}{[}\PY{n}{i}\PY{p}{]} \PY{o}{\PYZlt{}}\PY{o}{\PYZhy{}} \PY{n}{j}\PY{o}{\PYZhy{}}\PY{l+m+mi}{1}
            \PY{k}{break}
        \PY{p}{\PYZcb{}}
    \PY{p}{\PYZcb{}}
\PY{p}{\PYZcb{}}
\PY{n}{age\PYZus{}label} \PY{o}{\PYZlt{}}\PY{o}{\PYZhy{}} \PY{n}{c}\PY{p}{(}\PY{p}{)}
\PY{k}{for}\PY{p}{(}\PY{n}{i} \PY{o+ow}{in} \PY{l+m+mi}{1}\PY{p}{:}\PY{n}{length}\PY{p}{(}\PY{n}{age\PYZus{}categories}\PY{err}{\PYZdl{}}\PY{n}{Ages}\PY{err}{\PYZdl{}}\PY{n}{Min}\PY{p}{)}\PY{p}{)}\PY{p}{\PYZob{}}
    \PY{n}{age\PYZus{}label} \PY{o}{\PYZlt{}}\PY{o}{\PYZhy{}} \PY{n}{c}\PY{p}{(}\PY{n}{age\PYZus{}label}\PY{p}{,} \PY{n}{paste0}\PY{p}{(}\PY{n}{age\PYZus{}categories}\PY{err}{\PYZdl{}}\PY{n}{Ages}\PY{err}{\PYZdl{}}\PY{n}{Min}\PY{p}{[}\PY{n}{i}\PY{p}{]}\PY{p}{,}\PY{l+s+s2}{\PYZdq{}}\PY{l+s+s2}{\PYZhy{}}\PY{l+s+s2}{\PYZdq{}}\PY{p}{,}\PY{n}{age\PYZus{}categories}\PY{err}{\PYZdl{}}\PY{n}{Ages}\PY{err}{\PYZdl{}}\PY{n}{Max}\PY{p}{[}\PY{n}{i}\PY{p}{]}\PY{p}{)}\PY{p}{)}
\PY{p}{\PYZcb{}}
\end{Verbatim}
\end{tcolorbox}

    \begin{tcolorbox}[breakable, size=fbox, boxrule=1pt, pad at break*=1mm,colback=cellbackground, colframe=cellborder]
\prompt{In}{incolor}{23}{\boxspacing}
\begin{Verbatim}[commandchars=\\\{\}]
\PY{n}{income\PYZus{}categories} \PY{o}{=} \PY{n+nb}{list}\PY{p}{(}\PY{l+s+s2}{\PYZdq{}}\PY{l+s+s2}{Incomes}\PY{l+s+s2}{\PYZdq{}}\PY{o}{=}\PY{n+nb}{list}\PY{p}{(}\PY{l+s+s2}{\PYZdq{}}\PY{l+s+s2}{Min}\PY{l+s+s2}{\PYZdq{}}\PY{o}{=}\PY{n}{c}\PY{p}{(}\PY{l+m+mi}{0}\PY{p}{,}\PY{l+m+mi}{60001}\PY{p}{,}\PY{l+m+mi}{125001}\PY{p}{)}\PY{p}{,} \PY{l+s+s2}{\PYZdq{}}\PY{l+s+s2}{Max}\PY{l+s+s2}{\PYZdq{}}\PY{o}{=}\PY{n}{c}\PY{p}{(}\PY{l+m+mi}{60000}\PY{p}{,}\PY{l+m+mi}{125000}\PY{p}{,}\PY{l+m+mi}{200000}\PY{p}{)}\PY{p}{)}\PY{p}{)}
\PY{n}{income} \PY{o}{\PYZlt{}}\PY{o}{\PYZhy{}} \PY{n}{integer}\PY{p}{(}\PY{n}{rows}\PY{p}{)}
\PY{k}{for}\PY{p}{(}\PY{n}{i} \PY{o+ow}{in} \PY{l+m+mi}{1}\PY{p}{:}\PY{n}{nrow}\PY{p}{(}\PY{n}{csv}\PY{p}{)}\PY{p}{)}\PY{p}{\PYZob{}}
    \PY{k}{for}\PY{p}{(}\PY{n}{j} \PY{o+ow}{in} \PY{l+m+mi}{1}\PY{p}{:}\PY{n}{length}\PY{p}{(}\PY{n}{income\PYZus{}categories}\PY{err}{\PYZdl{}}\PY{n}{Incomes}\PY{err}{\PYZdl{}}\PY{n}{Min}\PY{p}{)}\PY{p}{)}\PY{p}{\PYZob{}}
        \PY{k}{if}\PY{p}{(}\PY{n}{csv}\PY{err}{\PYZdl{}}\PY{n}{Income}\PY{p}{[}\PY{n}{i}\PY{p}{]} \PY{o}{\PYZlt{}}\PY{o}{=} \PY{n}{income\PYZus{}categories}\PY{err}{\PYZdl{}}\PY{n}{Incomes}\PY{err}{\PYZdl{}}\PY{n}{Max}\PY{p}{[}\PY{n}{j}\PY{p}{]} \PY{o}{\PYZam{}}\PY{o}{\PYZam{}} \PY{n}{csv}\PY{err}{\PYZdl{}}\PY{n}{Income}\PY{p}{[}\PY{n}{i}\PY{p}{]} \PY{o}{\PYZgt{}}\PY{o}{=} \PY{n}{income\PYZus{}categories}\PY{err}{\PYZdl{}}\PY{n}{Incomes}\PY{err}{\PYZdl{}}\PY{n}{Min}\PY{p}{[}\PY{n}{j}\PY{p}{]}\PY{p}{)}\PY{p}{\PYZob{}}
            \PY{n}{income}\PY{p}{[}\PY{n}{i}\PY{p}{]} \PY{o}{\PYZlt{}}\PY{o}{\PYZhy{}} \PY{n}{j}\PY{o}{\PYZhy{}}\PY{l+m+mi}{1}
            \PY{k}{break}
        \PY{p}{\PYZcb{}}
    \PY{p}{\PYZcb{}}
\PY{p}{\PYZcb{}}
\PY{n}{income\PYZus{}label} \PY{o}{\PYZlt{}}\PY{o}{\PYZhy{}} \PY{n}{c}\PY{p}{(}\PY{p}{)}
\PY{k}{for}\PY{p}{(}\PY{n}{i} \PY{o+ow}{in} \PY{l+m+mi}{1}\PY{p}{:}\PY{n}{length}\PY{p}{(}\PY{n}{income\PYZus{}categories}\PY{err}{\PYZdl{}}\PY{n}{Incomes}\PY{err}{\PYZdl{}}\PY{n}{Min}\PY{p}{)}\PY{p}{)}\PY{p}{\PYZob{}}
    \PY{n}{income\PYZus{}label} \PY{o}{\PYZlt{}}\PY{o}{\PYZhy{}} \PY{n}{c}\PY{p}{(}\PY{n}{income\PYZus{}label}\PY{p}{,} \PY{n}{paste0}\PY{p}{(}\PY{n}{income\PYZus{}categories}\PY{err}{\PYZdl{}}\PY{n}{Incomes}\PY{err}{\PYZdl{}}\PY{n}{Min}\PY{p}{[}\PY{n}{i}\PY{p}{]}\PY{p}{,}\PY{l+s+s2}{\PYZdq{}}\PY{l+s+s2}{\PYZhy{}}\PY{l+s+s2}{\PYZdq{}}\PY{p}{,}\PY{n}{income\PYZus{}categories}\PY{err}{\PYZdl{}}\PY{n}{Incomes}\PY{err}{\PYZdl{}}\PY{n}{Max}\PY{p}{[}\PY{n}{i}\PY{p}{]}\PY{p}{)}\PY{p}{)}
\PY{p}{\PYZcb{}}
\end{Verbatim}
\end{tcolorbox}

    \hypertarget{preprocessing-data---extract-multi-categorical-variable-non-ordinal}{%
\paragraph{Preprocessing Data - Extract Multi Categorical Variable
(non-ordinal)}\label{preprocessing-data---extract-multi-categorical-variable-non-ordinal}}

    \begin{tcolorbox}[breakable, size=fbox, boxrule=1pt, pad at break*=1mm,colback=cellbackground, colframe=cellborder]
\prompt{In}{incolor}{24}{\boxspacing}
\begin{Verbatim}[commandchars=\\\{\}]
\PY{n}{df\PYZus{}c} \PY{o}{\PYZlt{}}\PY{o}{\PYZhy{}} \PY{k}{as}\PY{o}{.}\PY{n}{data}\PY{o}{.}\PY{n}{frame}\PY{p}{(}\PY{n}{table}\PY{p}{(}\PY{n}{csv}\PY{p}{[}\PY{p}{,}\PY{l+m+mi}{2}\PY{p}{]}\PY{p}{)}\PY{p}{)}
\PY{n}{city} \PY{o}{\PYZlt{}}\PY{o}{\PYZhy{}} \PY{n}{matrix}\PY{p}{(}\PY{l+m+mi}{0}\PY{p}{,} \PY{n}{nrow} \PY{o}{=} \PY{n}{nrow}\PY{p}{(}\PY{n}{csv}\PY{p}{)}\PY{p}{,} \PY{n}{ncol} \PY{o}{=} \PY{n}{nrow}\PY{p}{(}\PY{n}{df\PYZus{}c}\PY{p}{)}\PY{p}{)}
\PY{k}{for}\PY{p}{(}\PY{n}{i} \PY{o+ow}{in} \PY{l+m+mi}{1}\PY{p}{:}\PY{n}{nrow}\PY{p}{(}\PY{n}{csv}\PY{p}{)}\PY{p}{)}\PY{p}{\PYZob{}}
    \PY{k}{for}\PY{p}{(}\PY{n}{j} \PY{o+ow}{in} \PY{l+m+mi}{1}\PY{p}{:}\PY{n}{nrow}\PY{p}{(}\PY{n}{df\PYZus{}c}\PY{p}{)}\PY{p}{)}\PY{p}{\PYZob{}}
        \PY{k}{if}\PY{p}{(}\PY{n}{csv}\PY{err}{\PYZdl{}}\PY{n}{City}\PY{p}{[}\PY{n}{i}\PY{p}{]} \PY{o}{==} \PY{n}{df\PYZus{}c}\PY{err}{\PYZdl{}}\PY{n}{Var1}\PY{p}{[}\PY{n}{j}\PY{p}{]}\PY{p}{)}\PY{p}{\PYZob{}}
            \PY{n}{city}\PY{p}{[}\PY{n}{i}\PY{p}{,}\PY{n}{j}\PY{p}{]} \PY{o}{\PYZlt{}}\PY{o}{\PYZhy{}} \PY{l+m+mi}{1}
            \PY{k}{break}
        \PY{p}{\PYZcb{}}
    \PY{p}{\PYZcb{}}
\PY{p}{\PYZcb{}}
\PY{n}{city\PYZus{}label} \PY{o}{\PYZlt{}}\PY{o}{\PYZhy{}} \PY{n}{df\PYZus{}c}\PY{err}{\PYZdl{}}\PY{n}{Var1}
\end{Verbatim}
\end{tcolorbox}

    \hypertarget{create-processed-dataframe}{%
\paragraph{Create processed
DataFrame}\label{create-processed-dataframe}}

    \begin{tcolorbox}[breakable, size=fbox, boxrule=1pt, pad at break*=1mm,colback=cellbackground, colframe=cellborder]
\prompt{In}{incolor}{39}{\boxspacing}
\begin{Verbatim}[commandchars=\\\{\}]
\PY{n}{labels} \PY{o}{\PYZlt{}}\PY{o}{\PYZhy{}} \PY{n+nb}{list}\PY{p}{(}\PY{l+s+s2}{\PYZdq{}}\PY{l+s+s2}{City}\PY{l+s+s2}{\PYZdq{}}\PY{o}{=}\PY{n}{city\PYZus{}label}\PY{p}{,} \PY{l+s+s2}{\PYZdq{}}\PY{l+s+s2}{Gender}\PY{l+s+s2}{\PYZdq{}}\PY{o}{=}\PY{n}{gender\PYZus{}label}\PY{p}{,} \PY{l+s+s2}{\PYZdq{}}\PY{l+s+s2}{Age}\PY{l+s+s2}{\PYZdq{}}\PY{o}{=}\PY{n}{age\PYZus{}label}\PY{p}{,} \PY{l+s+s2}{\PYZdq{}}\PY{l+s+s2}{Income}\PY{l+s+s2}{\PYZdq{}}\PY{o}{=}\PY{n}{income\PYZus{}label}\PY{p}{,} \PY{l+s+s2}{\PYZdq{}}\PY{l+s+s2}{Illness}\PY{l+s+s2}{\PYZdq{}}\PY{o}{=}\PY{n}{illness\PYZus{}label}\PY{p}{)}
\PY{n}{df} \PY{o}{\PYZlt{}}\PY{o}{\PYZhy{}} \PY{k}{as}\PY{o}{.}\PY{n}{data}\PY{o}{.}\PY{n}{frame}\PY{p}{(}\PY{n+nb}{list}\PY{p}{(}\PY{l+s+s2}{\PYZdq{}}\PY{l+s+s2}{City}\PY{l+s+s2}{\PYZdq{}}\PY{o}{=}\PY{n}{city}\PY{p}{,} \PY{l+s+s2}{\PYZdq{}}\PY{l+s+s2}{Gender}\PY{l+s+s2}{\PYZdq{}}\PY{o}{=}\PY{n}{gender}\PY{p}{,} \PY{l+s+s2}{\PYZdq{}}\PY{l+s+s2}{Age}\PY{l+s+s2}{\PYZdq{}}\PY{o}{=}\PY{n}{age}\PY{p}{,} \PY{l+s+s2}{\PYZdq{}}\PY{l+s+s2}{Income}\PY{l+s+s2}{\PYZdq{}}\PY{o}{=}\PY{n}{income}\PY{p}{,} \PY{l+s+s2}{\PYZdq{}}\PY{l+s+s2}{Illness}\PY{l+s+s2}{\PYZdq{}}\PY{o}{=}\PY{n}{illness}\PY{p}{)}\PY{p}{)}
\end{Verbatim}
\end{tcolorbox}

    \begin{tcolorbox}[breakable, size=fbox, boxrule=1pt, pad at break*=1mm,colback=cellbackground, colframe=cellborder]
\prompt{In}{incolor}{40}{\boxspacing}
\begin{Verbatim}[commandchars=\\\{\}]
\PY{n}{df}\PY{p}{[}\PY{l+m+mi}{1}\PY{p}{:}\PY{l+m+mi}{6}\PY{p}{,}\PY{p}{]}
\end{Verbatim}
\end{tcolorbox}

    \begin{tabular}{r|llllllllllll}
 City.1 & City.2 & City.3 & City.4 & City.5 & City.6 & City.7 & City.8 & Gender & Age & Income & Illness\\
\hline
	 0 & 0 & 1 & 0 & 0 & 0 & 0 & 0 & 1 & 2 & 0 & 0\\
	 0 & 0 & 1 & 0 & 0 & 0 & 0 & 0 & 1 & 3 & 0 & 0\\
	 0 & 0 & 1 & 0 & 0 & 0 & 0 & 0 & 1 & 2 & 0 & 0\\
	 0 & 0 & 1 & 0 & 0 & 0 & 0 & 0 & 1 & 2 & 0 & 0\\
	 0 & 0 & 1 & 0 & 0 & 0 & 0 & 0 & 1 & 2 & 0 & 0\\
	 0 & 0 & 1 & 0 & 0 & 0 & 0 & 0 & 0 & 1 & 0 & 0\\
\end{tabular}


    
    lastly, we will reduce the size (cause of memory usage)

    \begin{tcolorbox}[breakable, size=fbox, boxrule=1pt, pad at break*=1mm,colback=cellbackground, colframe=cellborder]
\prompt{In}{incolor}{41}{\boxspacing}
\begin{Verbatim}[commandchars=\\\{\}]
\PY{n+nb}{print}\PY{p}{(}\PY{n}{paste0}\PY{p}{(}\PY{n}{nrow}\PY{p}{(}\PY{n}{df}\PY{p}{)}\PY{p}{,} \PY{l+s+s2}{\PYZdq{}}\PY{l+s+s2}{*}\PY{l+s+s2}{\PYZdq{}}\PY{p}{,} \PY{n}{ncol}\PY{p}{(}\PY{n}{df}\PY{p}{)}\PY{p}{)}\PY{p}{)}
\PY{n}{df} \PY{o}{\PYZlt{}}\PY{o}{\PYZhy{}} \PY{n}{df}\PY{p}{[}\PY{n}{sample}\PY{p}{(}\PY{n}{nrow}\PY{p}{(}\PY{n}{df}\PY{p}{)}\PY{p}{,} \PY{l+m+mi}{3000}\PY{p}{)}\PY{p}{,} \PY{p}{]}
\PY{n+nb}{print}\PY{p}{(}\PY{n}{paste0}\PY{p}{(}\PY{n}{nrow}\PY{p}{(}\PY{n}{df}\PY{p}{)}\PY{p}{,} \PY{l+s+s2}{\PYZdq{}}\PY{l+s+s2}{*}\PY{l+s+s2}{\PYZdq{}}\PY{p}{,} \PY{n}{ncol}\PY{p}{(}\PY{n}{df}\PY{p}{)}\PY{p}{)}\PY{p}{)}
\end{Verbatim}
\end{tcolorbox}

    \begin{Verbatim}[commandchars=\\\{\}]
[1] "150000*12"
[1] "3000*12"
    \end{Verbatim}

    \hypertarget{clustering-and-validation}{%
\subsubsection{Clustering And
Validation}\label{clustering-and-validation}}

    \hypertarget{principal-component-analysis}{%
\paragraph{Principal Component
Analysis}\label{principal-component-analysis}}

    We will not speak a lot of PCA since it's not the subject, but here, we
will use it to reduce the dimension space to two axis.

    \begin{tcolorbox}[breakable, size=fbox, boxrule=1pt, pad at break*=1mm,colback=cellbackground, colframe=cellborder]
\prompt{In}{incolor}{42}{\boxspacing}
\begin{Verbatim}[commandchars=\\\{\}]
\PY{n}{pca\PYZus{}df} \PY{o}{\PYZlt{}}\PY{o}{\PYZhy{}} \PY{n}{dudi}\PY{o}{.}\PY{n}{pca}\PY{p}{(}\PY{n}{df}\PY{p}{,} \PY{n}{center}\PY{o}{=}\PY{n}{TRUE}\PY{p}{,} \PY{n}{scale}\PY{o}{=}\PY{n}{TRUE}\PY{p}{,} \PY{n}{scannf} \PY{o}{=} \PY{n}{FALSE}\PY{p}{,} \PY{n}{nf} \PY{o}{=} \PY{l+m+mi}{2}\PY{p}{)}
\end{Verbatim}
\end{tcolorbox}

    \begin{tcolorbox}[breakable, size=fbox, boxrule=1pt, pad at break*=1mm,colback=cellbackground, colframe=cellborder]
\prompt{In}{incolor}{43}{\boxspacing}
\begin{Verbatim}[commandchars=\\\{\}]
\PY{n}{screeplot}\PY{p}{(}\PY{n}{pca\PYZus{}df}\PY{p}{,} \PY{n}{main} \PY{o}{=} \PY{l+s+s2}{\PYZdq{}}\PY{l+s+s2}{Screeplot \PYZhy{} Eigenvalues}\PY{l+s+s2}{\PYZdq{}}\PY{p}{)}
\end{Verbatim}
\end{tcolorbox}

    \begin{center}
    \adjustimage{max size={0.9\linewidth}{0.9\paperheight}}{output_103_0.png}
    \end{center}
    { \hspace*{\fill} \\}
    
    We clearly see that the dataset was generated since the inertia is quite
constant over all the initial dimensions (excepting the last one),
meaning that PCA is not as usefull as we can think. But we will still
rely on the two first axis for this report.

No matter what, using the elbow method we see that two axis are still a
good choice even if they only represent a small percentage of
information from the inital dataset.

    \hypertarget{kmeans---centroid-based-clustering}{%
\paragraph{Kmeans - Centroid-based
Clustering}\label{kmeans---centroid-based-clustering}}

    We will firstly select the best number of clusters using the Dun Index
and the elbow method.

    \begin{tcolorbox}[breakable, size=fbox, boxrule=1pt, pad at break*=1mm,colback=cellbackground, colframe=cellborder]
\prompt{In}{incolor}{44}{\boxspacing}
\begin{Verbatim}[commandchars=\\\{\}]
\PY{n}{fviz\PYZus{}nbclust}\PY{p}{(}\PY{n}{pca\PYZus{}df}\PY{err}{\PYZdl{}}\PY{n}{li}\PY{p}{,} \PY{n}{kmeans}\PY{p}{,} \PY{n}{method} \PY{o}{=} \PY{l+s+s2}{\PYZdq{}}\PY{l+s+s2}{wss}\PY{l+s+s2}{\PYZdq{}}\PY{p}{)} \PY{o}{+} \PY{n}{geom\PYZus{}vline}\PY{p}{(}\PY{n}{xintercept} \PY{o}{=} \PY{l+m+mi}{4}\PY{p}{,} \PY{n}{linetype} \PY{o}{=} \PY{l+m+mi}{2}\PY{p}{)} \PY{o}{+} 
\PY{n}{labs}\PY{p}{(}\PY{n}{subtitle} \PY{o}{=} \PY{l+s+s2}{\PYZdq{}}\PY{l+s+s2}{Elbow method}\PY{l+s+s2}{\PYZdq{}}\PY{p}{)}
\end{Verbatim}
\end{tcolorbox}

    \begin{center}
    \adjustimage{max size={0.9\linewidth}{0.9\paperheight}}{output_107_0.png}
    \end{center}
    { \hspace*{\fill} \\}
    
    \begin{tcolorbox}[breakable, size=fbox, boxrule=1pt, pad at break*=1mm,colback=cellbackground, colframe=cellborder]
\prompt{In}{incolor}{45}{\boxspacing}
\begin{Verbatim}[commandchars=\\\{\}]
\PY{n}{k\PYZus{}means} \PY{o}{=} \PY{n}{kmeans}\PY{p}{(}\PY{n}{pca\PYZus{}df}\PY{err}{\PYZdl{}}\PY{n}{li}\PY{p}{,}\PY{n}{centers}\PY{o}{=}\PY{l+m+mi}{4}\PY{p}{)}
\end{Verbatim}
\end{tcolorbox}

    \begin{tcolorbox}[breakable, size=fbox, boxrule=1pt, pad at break*=1mm,colback=cellbackground, colframe=cellborder]
\prompt{In}{incolor}{46}{\boxspacing}
\begin{Verbatim}[commandchars=\\\{\}]
\PY{n}{fviz\PYZus{}cluster}\PY{p}{(}\PY{n}{k\PYZus{}means}\PY{p}{,} \PY{n}{df}\PY{p}{,} \PY{n}{ellipse}\PY{o}{.}\PY{n}{type} \PY{o}{=} \PY{l+s+s2}{\PYZdq{}}\PY{l+s+s2}{norm}\PY{l+s+s2}{\PYZdq{}}\PY{p}{,} \PY{n}{geom}\PY{o}{=}\PY{l+s+s2}{\PYZdq{}}\PY{l+s+s2}{point}\PY{l+s+s2}{\PYZdq{}}\PY{p}{)} \PY{o}{+} \PY{n}{labs}\PY{p}{(}\PY{n}{title}\PY{o}{=} \PY{l+s+s2}{\PYZdq{}}\PY{l+s+s2}{K\PYZhy{}means with 4 clusters}\PY{l+s+s2}{\PYZdq{}}\PY{p}{)} \PY{o}{+} 
\PY{n}{xlab}\PY{p}{(}\PY{l+s+s2}{\PYZdq{}}\PY{l+s+s2}{PCA\PYZhy{}1}\PY{l+s+s2}{\PYZdq{}}\PY{p}{)} \PY{o}{+} \PY{n}{ylab}\PY{p}{(}\PY{l+s+s2}{\PYZdq{}}\PY{l+s+s2}{PCA\PYZhy{}2}\PY{l+s+s2}{\PYZdq{}}\PY{p}{)}
\end{Verbatim}
\end{tcolorbox}

    \begin{center}
    \adjustimage{max size={0.9\linewidth}{0.9\paperheight}}{output_109_0.png}
    \end{center}
    { \hspace*{\fill} \\}
    
    \begin{tcolorbox}[breakable, size=fbox, boxrule=1pt, pad at break*=1mm,colback=cellbackground, colframe=cellborder]
\prompt{In}{incolor}{47}{\boxspacing}
\begin{Verbatim}[commandchars=\\\{\}]
\PY{n}{qplot}\PY{p}{(}\PY{n}{pca\PYZus{}df}\PY{err}{\PYZdl{}}\PY{n}{li}\PY{p}{[}\PY{p}{,}\PY{l+m+mi}{1}\PY{p}{]}\PY{p}{,} \PY{n}{pca\PYZus{}df}\PY{err}{\PYZdl{}}\PY{n}{li}\PY{p}{[}\PY{p}{,}\PY{l+m+mi}{2}\PY{p}{]}\PY{p}{,} \PY{n}{colour} \PY{o}{=} \PY{n}{df}\PY{err}{\PYZdl{}}\PY{n}{Income}\PY{p}{)} \PY{o}{+} 
\PY{n}{labs}\PY{p}{(}\PY{n}{title}\PY{o}{=} \PY{l+s+s2}{\PYZdq{}}\PY{l+s+s2}{PCA Projection regarding Income}\PY{l+s+s2}{\PYZdq{}}\PY{p}{,} \PY{n}{color} \PY{o}{=} \PY{l+s+s2}{\PYZdq{}}\PY{l+s+s2}{Inc. Cat.}\PY{l+s+s2}{\PYZdq{}}\PY{p}{)} \PY{o}{+} 
\PY{n}{xlab}\PY{p}{(}\PY{l+s+s2}{\PYZdq{}}\PY{l+s+s2}{PCA\PYZhy{}1}\PY{l+s+s2}{\PYZdq{}}\PY{p}{)} \PY{o}{+} \PY{n}{ylab}\PY{p}{(}\PY{l+s+s2}{\PYZdq{}}\PY{l+s+s2}{PCA\PYZhy{}2}\PY{l+s+s2}{\PYZdq{}}\PY{p}{)}
\end{Verbatim}
\end{tcolorbox}

    \begin{center}
    \adjustimage{max size={0.9\linewidth}{0.9\paperheight}}{output_110_0.png}
    \end{center}
    { \hspace*{\fill} \\}
    
    Here we clearly see that the PCA choose the categorical variable income
as a major factor of information. Thus, we can see that the clustering
seems to also take it in account in some ways.

But this is not the only thing.

    \begin{tcolorbox}[breakable, size=fbox, boxrule=1pt, pad at break*=1mm,colback=cellbackground, colframe=cellborder]
\prompt{In}{incolor}{48}{\boxspacing}
\begin{Verbatim}[commandchars=\\\{\}]
\PY{n}{qplot}\PY{p}{(}\PY{n}{pca\PYZus{}df}\PY{err}{\PYZdl{}}\PY{n}{li}\PY{p}{[}\PY{p}{,}\PY{l+m+mi}{1}\PY{p}{]}\PY{p}{,} \PY{n}{pca\PYZus{}df}\PY{err}{\PYZdl{}}\PY{n}{li}\PY{p}{[}\PY{p}{,}\PY{l+m+mi}{2}\PY{p}{]}\PY{p}{,} \PY{n}{colour} \PY{o}{=} \PY{n}{df}\PY{err}{\PYZdl{}}\PY{n}{City}\PY{o}{.}\PY{l+m+mi}{6}\PY{p}{)} \PY{o}{+} 
\PY{n}{labs}\PY{p}{(}\PY{n}{title}\PY{o}{=} \PY{l+s+s2}{\PYZdq{}}\PY{l+s+s2}{PCA Projection regarding City.6}\PY{l+s+s2}{\PYZdq{}}\PY{p}{,} \PY{n}{color} \PY{o}{=} \PY{l+s+s2}{\PYZdq{}}\PY{l+s+s2}{City.6}\PY{l+s+s2}{\PYZdq{}}\PY{p}{)} \PY{o}{+} 
\PY{n}{xlab}\PY{p}{(}\PY{l+s+s2}{\PYZdq{}}\PY{l+s+s2}{PCA\PYZhy{}1}\PY{l+s+s2}{\PYZdq{}}\PY{p}{)} \PY{o}{+} \PY{n}{ylab}\PY{p}{(}\PY{l+s+s2}{\PYZdq{}}\PY{l+s+s2}{PCA\PYZhy{}2}\PY{l+s+s2}{\PYZdq{}}\PY{p}{)}
\end{Verbatim}
\end{tcolorbox}

    \begin{center}
    \adjustimage{max size={0.9\linewidth}{0.9\paperheight}}{output_112_0.png}
    \end{center}
    { \hspace*{\fill} \\}
    
    \begin{tcolorbox}[breakable, size=fbox, boxrule=1pt, pad at break*=1mm,colback=cellbackground, colframe=cellborder]
\prompt{In}{incolor}{49}{\boxspacing}
\begin{Verbatim}[commandchars=\\\{\}]
\PY{n}{qplot}\PY{p}{(}\PY{n}{pca\PYZus{}df}\PY{err}{\PYZdl{}}\PY{n}{li}\PY{p}{[}\PY{p}{,}\PY{l+m+mi}{1}\PY{p}{]}\PY{p}{,} \PY{n}{pca\PYZus{}df}\PY{err}{\PYZdl{}}\PY{n}{li}\PY{p}{[}\PY{p}{,}\PY{l+m+mi}{2}\PY{p}{]}\PY{p}{,} \PY{n}{colour} \PY{o}{=} \PY{n}{df}\PY{err}{\PYZdl{}}\PY{n}{City}\PY{o}{.}\PY{l+m+mi}{5}\PY{p}{)} \PY{o}{+} 
\PY{n}{labs}\PY{p}{(}\PY{n}{title}\PY{o}{=} \PY{l+s+s2}{\PYZdq{}}\PY{l+s+s2}{PCA Projection regarding City.5}\PY{l+s+s2}{\PYZdq{}}\PY{p}{,} \PY{n}{color} \PY{o}{=} \PY{l+s+s2}{\PYZdq{}}\PY{l+s+s2}{City.5}\PY{l+s+s2}{\PYZdq{}}\PY{p}{)} \PY{o}{+} 
\PY{n}{xlab}\PY{p}{(}\PY{l+s+s2}{\PYZdq{}}\PY{l+s+s2}{PCA\PYZhy{}1}\PY{l+s+s2}{\PYZdq{}}\PY{p}{)} \PY{o}{+} \PY{n}{ylab}\PY{p}{(}\PY{l+s+s2}{\PYZdq{}}\PY{l+s+s2}{PCA\PYZhy{}2}\PY{l+s+s2}{\PYZdq{}}\PY{p}{)}
\end{Verbatim}
\end{tcolorbox}

    \begin{center}
    \adjustimage{max size={0.9\linewidth}{0.9\paperheight}}{output_113_0.png}
    \end{center}
    { \hspace*{\fill} \\}
    
    Here we see that Cities introduce a huge bias in our PCA. The other
variables are all well distributed in the clusters. So we will try to
delete cities from our pca.

\begin{center}\rule{0.5\linewidth}{0.5pt}\end{center}

Note that something is still interesting :

\begin{itemize}
\tightlist
\item
  As we can see in the city.6, the richer people seems to be
  concentrated, and the clustering seems to have understand this.
\end{itemize}

    \begin{tcolorbox}[breakable, size=fbox, boxrule=1pt, pad at break*=1mm,colback=cellbackground, colframe=cellborder]
\prompt{In}{incolor}{50}{\boxspacing}
\begin{Verbatim}[commandchars=\\\{\}]
\PY{n}{df} \PY{o}{\PYZlt{}}\PY{o}{\PYZhy{}} \PY{n}{df}\PY{p}{[}\PY{p}{,}\PY{p}{(}\PY{n}{ncol}\PY{p}{(}\PY{n}{df}\PY{p}{)}\PY{o}{\PYZhy{}}\PY{l+m+mi}{3}\PY{p}{)}\PY{p}{:}\PY{n}{ncol}\PY{p}{(}\PY{n}{df}\PY{p}{)}\PY{p}{]}
\PY{n}{head}\PY{p}{(}\PY{n}{df}\PY{p}{)}
\end{Verbatim}
\end{tcolorbox}

    \begin{tabular}{r|llll}
  & Gender & Age & Income & Illness\\
\hline
	134058 & 0 & 2 & 1 & 0\\
	124022 & 0 & 2 & 1 & 0\\
	103065 & 0 & 4 & 2 & 1\\
	124507 & 1 & 3 & 1 & 0\\
	45404 & 0 & 2 & 1 & 0\\
	65161 & 0 & 2 & 1 & 0\\
\end{tabular}


    
    \begin{tcolorbox}[breakable, size=fbox, boxrule=1pt, pad at break*=1mm,colback=cellbackground, colframe=cellborder]
\prompt{In}{incolor}{51}{\boxspacing}
\begin{Verbatim}[commandchars=\\\{\}]
\PY{n}{pca\PYZus{}df} \PY{o}{\PYZlt{}}\PY{o}{\PYZhy{}} \PY{n}{dudi}\PY{o}{.}\PY{n}{pca}\PY{p}{(}\PY{n}{df}\PY{p}{,} \PY{n}{center}\PY{o}{=}\PY{n}{TRUE}\PY{p}{,} \PY{n}{scale}\PY{o}{=}\PY{n}{TRUE}\PY{p}{,} \PY{n}{scannf} \PY{o}{=} \PY{n}{FALSE}\PY{p}{,} \PY{n}{nf} \PY{o}{=} \PY{l+m+mi}{2}\PY{p}{)}
\end{Verbatim}
\end{tcolorbox}

    \begin{tcolorbox}[breakable, size=fbox, boxrule=1pt, pad at break*=1mm,colback=cellbackground, colframe=cellborder]
\prompt{In}{incolor}{52}{\boxspacing}
\begin{Verbatim}[commandchars=\\\{\}]
\PY{n}{fviz\PYZus{}nbclust}\PY{p}{(}\PY{n}{pca\PYZus{}df}\PY{err}{\PYZdl{}}\PY{n}{li}\PY{p}{,} \PY{n}{kmeans}\PY{p}{,} \PY{n}{method} \PY{o}{=} \PY{l+s+s2}{\PYZdq{}}\PY{l+s+s2}{wss}\PY{l+s+s2}{\PYZdq{}}\PY{p}{)} \PY{o}{+} \PY{n}{geom\PYZus{}vline}\PY{p}{(}\PY{n}{xintercept} \PY{o}{=} \PY{l+m+mi}{5}\PY{p}{,} \PY{n}{linetype} \PY{o}{=} \PY{l+m+mi}{2}\PY{p}{)}
\PY{o}{+} \PY{n}{labs}\PY{p}{(}\PY{n}{subtitle} \PY{o}{=} \PY{l+s+s2}{\PYZdq{}}\PY{l+s+s2}{Elbow method}\PY{l+s+s2}{\PYZdq{}}\PY{p}{)}
\end{Verbatim}
\end{tcolorbox}

    \begin{Verbatim}[commandchars=\\\{\}, frame=single, framerule=2mm, rulecolor=\color{outerrorbackground}]
Error in +labs(subtitle = "Elbow method"): argument incorrect pour un opérateur unitaire
Traceback:

    \end{Verbatim}

    \begin{center}
    \adjustimage{max size={0.9\linewidth}{0.9\paperheight}}{output_117_1.png}
    \end{center}
    { \hspace*{\fill} \\}
    
    \begin{tcolorbox}[breakable, size=fbox, boxrule=1pt, pad at break*=1mm,colback=cellbackground, colframe=cellborder]
\prompt{In}{incolor}{53}{\boxspacing}
\begin{Verbatim}[commandchars=\\\{\}]
\PY{n}{k\PYZus{}means} \PY{o}{=} \PY{n}{kmeans}\PY{p}{(}\PY{n}{pca\PYZus{}df}\PY{err}{\PYZdl{}}\PY{n}{li}\PY{p}{,}\PY{n}{centers}\PY{o}{=}\PY{l+m+mi}{5}\PY{p}{)}
\end{Verbatim}
\end{tcolorbox}

    \begin{tcolorbox}[breakable, size=fbox, boxrule=1pt, pad at break*=1mm,colback=cellbackground, colframe=cellborder]
\prompt{In}{incolor}{54}{\boxspacing}
\begin{Verbatim}[commandchars=\\\{\}]
\PY{n}{fviz\PYZus{}cluster}\PY{p}{(}\PY{n}{k\PYZus{}means}\PY{p}{,} \PY{n}{df}\PY{p}{,} \PY{n}{ellipse}\PY{o}{.}\PY{n}{type} \PY{o}{=} \PY{l+s+s2}{\PYZdq{}}\PY{l+s+s2}{norm}\PY{l+s+s2}{\PYZdq{}}\PY{p}{,} \PY{n}{geom}\PY{o}{=}\PY{l+s+s2}{\PYZdq{}}\PY{l+s+s2}{point}\PY{l+s+s2}{\PYZdq{}}\PY{p}{)} \PY{o}{+} \PY{n}{labs}\PY{p}{(}\PY{n}{title}\PY{o}{=} \PY{l+s+s2}{\PYZdq{}}\PY{l+s+s2}{K\PYZhy{}means with 5 clusters}\PY{l+s+s2}{\PYZdq{}}\PY{p}{)} \PY{o}{+} 
\PY{n}{xlab}\PY{p}{(}\PY{l+s+s2}{\PYZdq{}}\PY{l+s+s2}{PCA\PYZhy{}1}\PY{l+s+s2}{\PYZdq{}}\PY{p}{)} \PY{o}{+} \PY{n}{ylab}\PY{p}{(}\PY{l+s+s2}{\PYZdq{}}\PY{l+s+s2}{PCA\PYZhy{}2}\PY{l+s+s2}{\PYZdq{}}\PY{p}{)}
\end{Verbatim}
\end{tcolorbox}

    \begin{center}
    \adjustimage{max size={0.9\linewidth}{0.9\paperheight}}{output_119_0.png}
    \end{center}
    { \hspace*{\fill} \\}
    
    \begin{tcolorbox}[breakable, size=fbox, boxrule=1pt, pad at break*=1mm,colback=cellbackground, colframe=cellborder]
\prompt{In}{incolor}{55}{\boxspacing}
\begin{Verbatim}[commandchars=\\\{\}]
\PY{n}{qplot}\PY{p}{(}\PY{n}{pca\PYZus{}df}\PY{err}{\PYZdl{}}\PY{n}{li}\PY{p}{[}\PY{p}{,}\PY{l+m+mi}{1}\PY{p}{]}\PY{p}{,} \PY{n}{pca\PYZus{}df}\PY{err}{\PYZdl{}}\PY{n}{li}\PY{p}{[}\PY{p}{,}\PY{l+m+mi}{2}\PY{p}{]}\PY{p}{,} \PY{n}{colour} \PY{o}{=} \PY{n}{df}\PY{err}{\PYZdl{}}\PY{n}{Illness}\PY{p}{)} \PY{o}{+} 
\PY{n}{labs}\PY{p}{(}\PY{n}{title}\PY{o}{=} \PY{l+s+s2}{\PYZdq{}}\PY{l+s+s2}{PCA Projection regarding Illness}\PY{l+s+s2}{\PYZdq{}}\PY{p}{,} \PY{n}{color} \PY{o}{=} \PY{l+s+s2}{\PYZdq{}}\PY{l+s+s2}{Illness}\PY{l+s+s2}{\PYZdq{}}\PY{p}{)} \PY{o}{+} 
\PY{n}{xlab}\PY{p}{(}\PY{l+s+s2}{\PYZdq{}}\PY{l+s+s2}{PCA\PYZhy{}1}\PY{l+s+s2}{\PYZdq{}}\PY{p}{)} \PY{o}{+} \PY{n}{ylab}\PY{p}{(}\PY{l+s+s2}{\PYZdq{}}\PY{l+s+s2}{PCA\PYZhy{}2}\PY{l+s+s2}{\PYZdq{}}\PY{p}{)}
\end{Verbatim}
\end{tcolorbox}

    \begin{center}
    \adjustimage{max size={0.9\linewidth}{0.9\paperheight}}{output_120_0.png}
    \end{center}
    { \hspace*{\fill} \\}
    
    \begin{tcolorbox}[breakable, size=fbox, boxrule=1pt, pad at break*=1mm,colback=cellbackground, colframe=cellborder]
\prompt{In}{incolor}{56}{\boxspacing}
\begin{Verbatim}[commandchars=\\\{\}]
\PY{n}{qplot}\PY{p}{(}\PY{n}{pca\PYZus{}df}\PY{err}{\PYZdl{}}\PY{n}{li}\PY{p}{[}\PY{p}{,}\PY{l+m+mi}{1}\PY{p}{]}\PY{p}{,} \PY{n}{pca\PYZus{}df}\PY{err}{\PYZdl{}}\PY{n}{li}\PY{p}{[}\PY{p}{,}\PY{l+m+mi}{2}\PY{p}{]}\PY{p}{,} \PY{n}{colour} \PY{o}{=} \PY{n}{df}\PY{err}{\PYZdl{}}\PY{n}{Income}\PY{p}{)} \PY{o}{+} 
\PY{n}{labs}\PY{p}{(}\PY{n}{title}\PY{o}{=} \PY{l+s+s2}{\PYZdq{}}\PY{l+s+s2}{PCA Projection regarding Income}\PY{l+s+s2}{\PYZdq{}}\PY{p}{,} \PY{n}{color} \PY{o}{=} \PY{l+s+s2}{\PYZdq{}}\PY{l+s+s2}{Inc. Cat.}\PY{l+s+s2}{\PYZdq{}}\PY{p}{)} \PY{o}{+} 
\PY{n}{xlab}\PY{p}{(}\PY{l+s+s2}{\PYZdq{}}\PY{l+s+s2}{PCA\PYZhy{}1}\PY{l+s+s2}{\PYZdq{}}\PY{p}{)} \PY{o}{+} \PY{n}{ylab}\PY{p}{(}\PY{l+s+s2}{\PYZdq{}}\PY{l+s+s2}{PCA\PYZhy{}2}\PY{l+s+s2}{\PYZdq{}}\PY{p}{)}
\end{Verbatim}
\end{tcolorbox}

    \begin{center}
    \adjustimage{max size={0.9\linewidth}{0.9\paperheight}}{output_121_0.png}
    \end{center}
    { \hspace*{\fill} \\}
    
    \begin{tcolorbox}[breakable, size=fbox, boxrule=1pt, pad at break*=1mm,colback=cellbackground, colframe=cellborder]
\prompt{In}{incolor}{57}{\boxspacing}
\begin{Verbatim}[commandchars=\\\{\}]
\PY{n}{qplot}\PY{p}{(}\PY{n}{pca\PYZus{}df}\PY{err}{\PYZdl{}}\PY{n}{li}\PY{p}{[}\PY{p}{,}\PY{l+m+mi}{1}\PY{p}{]}\PY{p}{,} \PY{n}{pca\PYZus{}df}\PY{err}{\PYZdl{}}\PY{n}{li}\PY{p}{[}\PY{p}{,}\PY{l+m+mi}{2}\PY{p}{]}\PY{p}{,} \PY{n}{colour} \PY{o}{=} \PY{n}{df}\PY{err}{\PYZdl{}}\PY{n}{Gender}\PY{p}{)} \PY{o}{+} 
\PY{n}{labs}\PY{p}{(}\PY{n}{title}\PY{o}{=} \PY{l+s+s2}{\PYZdq{}}\PY{l+s+s2}{PCA Projection regarding Gender}\PY{l+s+s2}{\PYZdq{}}\PY{p}{,} \PY{n}{color} \PY{o}{=} \PY{l+s+s2}{\PYZdq{}}\PY{l+s+s2}{Gender}\PY{l+s+s2}{\PYZdq{}}\PY{p}{)} \PY{o}{+} 
\PY{n}{xlab}\PY{p}{(}\PY{l+s+s2}{\PYZdq{}}\PY{l+s+s2}{PCA\PYZhy{}1}\PY{l+s+s2}{\PYZdq{}}\PY{p}{)} \PY{o}{+} \PY{n}{ylab}\PY{p}{(}\PY{l+s+s2}{\PYZdq{}}\PY{l+s+s2}{PCA\PYZhy{}2}\PY{l+s+s2}{\PYZdq{}}\PY{p}{)}
\end{Verbatim}
\end{tcolorbox}

    \begin{center}
    \adjustimage{max size={0.9\linewidth}{0.9\paperheight}}{output_122_0.png}
    \end{center}
    { \hspace*{\fill} \\}
    
    \begin{tcolorbox}[breakable, size=fbox, boxrule=1pt, pad at break*=1mm,colback=cellbackground, colframe=cellborder]
\prompt{In}{incolor}{58}{\boxspacing}
\begin{Verbatim}[commandchars=\\\{\}]
\PY{n}{qplot}\PY{p}{(}\PY{n}{pca\PYZus{}df}\PY{err}{\PYZdl{}}\PY{n}{li}\PY{p}{[}\PY{p}{,}\PY{l+m+mi}{1}\PY{p}{]}\PY{p}{,} \PY{n}{pca\PYZus{}df}\PY{err}{\PYZdl{}}\PY{n}{li}\PY{p}{[}\PY{p}{,}\PY{l+m+mi}{2}\PY{p}{]}\PY{p}{,} \PY{n}{colour} \PY{o}{=} \PY{n}{df}\PY{err}{\PYZdl{}}\PY{n}{Age}\PY{p}{)} \PY{o}{+} 
\PY{n}{labs}\PY{p}{(}\PY{n}{title}\PY{o}{=} \PY{l+s+s2}{\PYZdq{}}\PY{l+s+s2}{PCA Projection regarding Age}\PY{l+s+s2}{\PYZdq{}}\PY{p}{,} \PY{n}{color} \PY{o}{=} \PY{l+s+s2}{\PYZdq{}}\PY{l+s+s2}{Age}\PY{l+s+s2}{\PYZdq{}}\PY{p}{)} \PY{o}{+} 
\PY{n}{xlab}\PY{p}{(}\PY{l+s+s2}{\PYZdq{}}\PY{l+s+s2}{PCA\PYZhy{}1}\PY{l+s+s2}{\PYZdq{}}\PY{p}{)} \PY{o}{+} \PY{n}{ylab}\PY{p}{(}\PY{l+s+s2}{\PYZdq{}}\PY{l+s+s2}{PCA\PYZhy{}2}\PY{l+s+s2}{\PYZdq{}}\PY{p}{)}
\end{Verbatim}
\end{tcolorbox}

    \begin{center}
    \adjustimage{max size={0.9\linewidth}{0.9\paperheight}}{output_123_0.png}
    \end{center}
    { \hspace*{\fill} \\}
    
    In this case, the clustering is not really good. But it still better
than before :

\begin{itemize}
\tightlist
\item
  We see that the PCA decided to create to distinct cluster based on
  illness which in our case could be interessant.
\item
  Data are also clustered by the income category.
\end{itemize}

We could say those basics laws regarding to the result of clustering :

\begin{itemize}
\tightlist
\item
  Violet Cluster =\textgreater{} Non-ill
\item
  The other are ill, but with different criteria based on income, age or
  gender
\end{itemize}

    \hypertarget{bootstraping-validation}{%
\paragraph{Bootstraping Validation}\label{bootstraping-validation}}

    \begin{tcolorbox}[breakable, size=fbox, boxrule=1pt, pad at break*=1mm,colback=cellbackground, colframe=cellborder]
\prompt{In}{incolor}{59}{\boxspacing}
\begin{Verbatim}[commandchars=\\\{\}]
\PY{n}{k}\PY{o}{.}\PY{n}{select}\PY{p}{(}\PY{n}{pca\PYZus{}df}\PY{err}{\PYZdl{}}\PY{n}{li}\PY{p}{,} \PY{l+m+mi}{2}\PY{p}{:}\PY{l+m+mi}{8}\PY{p}{,} \PY{n}{B} \PY{o}{=} \PY{l+m+mi}{10}\PY{p}{,} \PY{n}{r} \PY{o}{=} \PY{l+m+mi}{3}\PY{p}{,} \PY{n}{threshold} \PY{o}{=} \PY{l+m+mf}{0.6}\PY{p}{,} \PY{n}{scheme\PYZus{}2} \PY{o}{=} \PY{n}{TRUE}\PY{p}{)}
\end{Verbatim}
\end{tcolorbox}

    \begin{description}
\item[\$profile] \begin{description*}
\item[2] 0.801903234295317
\item[3] 0.592810080143077
\item[4] 0.422272936659293
\item[5] 0.224290352705193
\item[6] 0.234737062584996
\item[7] 0.390443069880361
\item[8] 0.25228254304847
\end{description*}

\item[\$k] 2
\end{description}


    
    As we may have already find, the best k in with the pca is 2. Meaning
that the best cluster would be Ill and Non-Ill.


    % Add a bibliography block to the postdoc
    
    
    
\end{document}
